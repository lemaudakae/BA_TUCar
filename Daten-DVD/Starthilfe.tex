\documentclass[
%	draft,			% Entwurfsmodus: Bilder als Rahmen,
	11pt,			% KOMA default
	a4paper,		% DIN A4
	oneside,		% Zweiseitig (oneside für einseitiges Layout)
	german,			%
	titlepage		%
]{scrartcl}			% siehe <http://www.komascript.de>

\usepackage{selinput}		% Eingabecodierung automatisch ermitteln …
\SelectInputMappings{		% … siehe <http://ctan.org/pkg/selinput>
  adieresis={ä},
  germandbls={ß},
}
\usepackage[ngerman]{babel}
\usepackage[onehalfspacing]{setspace}		% 1.5-facher Zeilenabstand
\usepackage{minted} %[cache=false]

\newcommand{\sh}[1]{\mint[xleftmargin=2em]{shell}{#1}}
%\newmint[sh]{shell}{xleftmargin=2em}
%\newmint{shell}{xleftmargin=2em}
\newminted[she]{shell}{xleftmargin=2em}
\newmintinline[shi]{shell}{}
\title{\Huge
	Bedienungsanleitung\\ \LARGE Fahrspurverfolgung TUCar \\   
}
\author{Leopold Mauersberger}
\date{\today}


\begin{document}

% Titel
\maketitle

\section{Gültigkeitsbereich \& Voraussetzungen}
Diese Anleitung beschreibt den Start der im Rahmen der Bachelorarbeit \glqq Kamerabasierte Navigation eines Modellfahrzeugs in einem Straßenverkehrsszenario\grqq\ implementierten Fahrspurverfolgung. Für ein weiterführendes Verständnis der hier beschrieben Vorgehensweisen sei auf das ROS-Wiki \footnote{http://wiki.ros.org/} sowie die MATLAB-Dokumentation, speziell die Abschnitte der Robotics-System-Toolbox \footnote{https://de.mathworks.com/products/robotics.html}, verwiesen. Zudem wird die Fähigkeit mit einem Linux-Terminal zu arbeiten, vorausgesetzt.

\section{Notwendige Verbindungen}
Bevor die entsprechenden Applikationen gestarten werden können, muss der PC sich auf dem vom Raspberry PI bereitgestellten WLAN-Hotspot einloggen. Das benötigte Passwort liegt dem Fahrzeug bei oder ist beim Admin zu erfragen. Zudem empfiehlt es sich, kurz ein LAN-Kabel am Raspberry PI anzustecken, um die Zeit des Einplatinencomputers ohne Echtzeituhr einzustellen. Einige generierte Debugging-Daten besitzen andernfalls keine Aussagekraft.

\section{Raspberry PI}
Ist die Wlan-Verbindung zwischen PC und PI hergestellt, kann sich in einer neuen Konsole via
\sh{ssh 10.0.0.1}
%\mint[xleftmargin=5em]{shell}{ssh lemau@10.0.0.1} 
\noindent auf diesem eingeloggt werden.
\noindent Bei erstmaligem Start installieren die folgenden Befehle die notwendigen Nodes auf dem Raspi.
\begin{she}
# ROS-Befehle verfügbar machen
. /opt/ros/kinetic/setup.bash
# Ordner für Nodes anlegen
mkdir tucar_ws
cd tucar_ws
# Installationsskript holen & ausführen
svn checkout 
https://borstel.etit.tu-chemnitz.de/svn/praktikum/TUCar/trunk/ROS/rosinstall
cd rosinstall
./install.sh
# Nodes bauen
cd ../catkin
catkin_make
\end{she}
Jetzt (auch bei jedem späteren Start) müssen die gebauten Nodes bekannt gemacht (gesourced) werden.
\sh{source ~/tucar_ws/catkin/devel/setup.bash}
\noindent Danach sollte die IP-Adresse des Netzwerkinterfaces, welches ROS verwenden soll, festgelegt werden.
\sh{export ROS_HOSTNAME=10.0.0.1}
\noindent Nun kann das Launchfile zum Starten aller notwendigen Nodes aufgerufen werden:
\sh{roslaunch tucar tucar_bringup.launch use_sensors:=true}
\noindent Beendet werden diese durch die Tastenkombination \framebox{STRG} + \framebox{C}. Der PI kann mittels
\sh{Poweroff} 
\noindent heruntergefahren werden.

\subsection{bashrc}
Der Einfachheit können die Befehle
\begin{she}
. /opt/ros/kinetic/setup.bash
source ~/tucar_ws/catkin/devel/setup.bash
export ROS_HOSTNAME=10.0.0.1
\end{she}
\noindent in die \emph{bashrc} des PI eingetragen werden, um die Ausführung bei jedem Start automatisch zu veranlassen.

\section{PC}
\subsection{Matlab}
\label{ssec:pc:matlab}
Am PC kann nun MATLAB gestartet werden. Hierbei ist es wichtig, dies aus dem Ordner \emph{Punktbasierte Fahrspurerkennung} zu tun, da sich hier das Skript \emph{startup.m} befindet, welches die benötigten Custom-Messages des TUCars lädt, so wie alle relevanten Unterordner dem Matlab-Pfad hinzufügt:
\begin{she}
cd /Pfad/zur/DVD/Punktbasierte Fahrspurerkennung
matlab&
\end{she}
Bei erstmaligem Test des TUCars muss eine Textzeile (Pfadangabe) in eine bestimmte (meist noch zu erstellende) Matlab-Config-Datei eingefügt werden. Die Anweisung dazu taucht in der Kommandozeile Matlabs auf. 

Jetzt muss auch das von ROS am PC zu nutzende Netzwerk-Interface festgelegt werden. Dies geschieht beim Start des Matlab-Skriptes automatisch. Es kann jedoch notwendig sein, diese IP anzupassen:
\sh{ifconfig} 
\noindent in einem weiteren Terminal aufgerufen liefert Informationen zu allen vorhandenen Netzwerkadaptern, der WLAN-Stick sollte hier ausfindig zu machen sein.
Die gefundene Adresse kann nun gegebenenfalls in der Datei \emph{function\_init\_parameters.m} im Ordner \emph{Initialisierungen} unter Punkt \emph{ROS-Parameter} bei \emph{params.RosIP} eingetragen werden.

Zum Start der Fahrspurverfolgung muss lediglich die Datei \emph{main.m} ausgeführt werden. Die Debuggingmöglichkeiten können von hier aus eingeschaltet werden, genaue Erläuterungen finden sich in den Kommentaren. Zum Auswerten dieser sind die Skripte im Ordner \emph{Debug} zu nutzen. Die Arbeitsweise des Quellcodes kann nach dem Lesen der zugehörigen Bachelorarbeit anhand der Funktionsbeschreibungen durch rekursives Abarbeiten dieser ergründet werden. Beendet wird die Fahrspurverfolgung durch einen Klick in die geöffnete, leere Figure oder das Drücken einer beliebigen Taste.

\subsection{Konsole}
Soll nicht nur via Matlab mit dem TUCar kommuniziert werden, so muss ein weiteres Terminal geöffnet werden, in dem zum Nutzen der Custom-Messages ebenfalls der TUCar-Node installiert und gesourced werden muss.
\begin{she}
# ROS-Befehle verfügbar machen
. /opt/ros/kinetic/setup.bash
# Ordner für Nodes anlegen
mkdir tucar_ws
cd tucar_ws
# Installationsskript holen & ausführen
svn checkout 
https://borstel.etit.tu-chemnitz.de/svn/praktikum/TUCar/trunk/ROS/rosinstall
cd rosinstall
./install.sh
# Nodes bauen
cd ../catkin
catkin_make
# Nodes sourcen
source ~/tucar_ws/catkin/devel/setup.bash
\end{she}

\noindent Desweiteren sollte die IP des ROS-Masters (PI) bekannt gemacht und die richtige Netzwerkschnittstelle eingestellt werden.
\begin{she}
export ROS_MASTER_URI=http://10.0.0.1:11311 # IP PI-Hotspot
export ROS_IP=x.x.x.x # IP Wlan-Stick -> siehe ifconfig
\end{she}

\noindent Eine kleine Liste nützlicher Befehle ist:
\begin{she}
# verfügbare Topics anzeigen
rostopic list
# Topicdaten auf Konsole ausgeben
rostopic echo topicname
# Kamerabild anzeigen
rqt_image_view
\end{she}

\subsubsection{Sondereinstellungen Polynombasierte Fahrspurerkennung}
Zum Nutzen des Rosbags sind folgende Befehle auszuführen (jeweils in seperatem Terminal):
\begin{she}
# roscore starten
roscore
# Topicdaten auf Konsole ausgeben
rosbag play -l <Dateiname>.bag
# Kamerabild anzeigen
rqt_image_view
# Throttle-Node zum Senken der Bildfrequenz starten
rosrun topic_tools throttle messages /tucar/omnicam/full/compressed_throttle <Bildfrequenz>
\end{she}

In Matlab muss neben der in Abschnitt \ref{ssec:pc:matlab} beschriebenen Änderung der IP des zu nutzenden Netzwerkinterfaces auch die IP des ROS-Masters auf die LAN-Schnittstelle des Rechners festgelegt werden. Der anzupassende Parameter lautet \emph{params.RosMasterUri}.

\section{Parameteranpassung TUCar}
\subsection{Fahrplattform}
Die Konfiguration der Fahrplattform kann mittels der Datei 
\sh{tucar_controller.yaml} 
\noindent im Ordner 
\sh{~/tucar_ws/catkin/src/tucar/tucar/config/platform} 
\noindent erfolgen. Hier kann z.B. die Frequenz der Odometrie- und IMU-Datenpakete reguliert werden. 
\subsection{Kamera}
\noindent Die Datei 
\sh{camera.xml} 
\noindent im Ordner 
\sh{~/tucar_ws/catkin/src/tucar/tucar/launch/sensor_drivers} 
\noindent ist für die Konfiguration der Kamera zuständig. Hier kann z.B. Auflösung und Framerate eingestellt werden.

\end{document}
