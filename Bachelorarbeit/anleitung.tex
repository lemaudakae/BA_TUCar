\chapter{Einleitung}
Diese Datei enthält die Anleitung zur Nutzung der Vorlage für verschiedene
Typen von Arbeiten. Sie ist vorrangig für Studenten (und auch wissenschaftliche
Mitarbeiter) gedacht, welche ihre Arbeiten bzw. Publikationen mit \LaTeX{}
erstellen wollen. Dabei wurden auch die Richtlinien des Corporate Design
der Technischen Universität Chemnitz berücksichtigt, soweit sie sich
ohne größere Probleme in \LaTeX{} realisieren lassen.

Diese Vorlage ist für folgende Dokumente konzipiert, kann aber bei geringen
Modifikationen auch darüber hinaus eingesetzt werden:
\begin{itemize}
\item Hausarbeiten
\item Studienarbeiten
\item Diplomarbeiten
\item Praktikumsberichte
\item Proseminare
\item Oberseminare
\item Hauptseminare
\item sonstige Seminare
\item Belege
\item Studien
\end{itemize}

In den folgenden Kapiteln dieser Anleitung wird ein Überblick über die
Verwendung der Vorlage, Zweck der Dateien und typische Anwendungsfälle
gegeben. 

\textbf{Hinweis:} Diese Anleitung ist \textbf{keine} Einführung
in \LaTeX. Dazu sei auf das Kursangebot des URZ bzw. auf weiterführende
Literatur verwiesen. Auch erhebt diese Vorlage \textbf{nicht} den Anspruch,
daß jedes damit erstellte Dokument innerhalb der TU-Chemnitz grundsätzlich
in Form, Umfang und Aufbau anerkannt wird. Studenten sollten dies 
grundsätzlich vor der Verwendung anhand der für sie gültigen Studien- 
und Prüfungsordnung prüfen und darüber hinaus mit dem für sie 
zuständigen Professor bzw. Betreuer klären.

\section{Grundkonzept}
Alle auf Basis dieser Vorlage erstellten Dokumente verwenden das sogenannte
{\scshape Koma-Script}-Paket. Dieses Paket wurde entwickelt um \LaTeX{} den
europäischen Anforderungen (insbesondere Deutschland) anzupassen.
Diese Anforderungen umfassen u.a.
\begin{itemize}
\item Papierformate
\item verschiedene Sprachen
\item verschiedene Datumsformate
\end{itemize}
Für weitere Details sei auf die Dokumentation zu {\scshape Koma-Script} 
verwiesen.


