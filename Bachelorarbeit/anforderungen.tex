\chapter{Anforderungen und Hinweise der Professur}
Allgemeine Hinweise zur Anfertigung einer wissenschaftlichen Arbeit an der Professur Prozessautomatisierung sind unter folgendem Link zu finden: \url{https://www.tu-chemnitz.de/etit/proaut/lehrmaterial/allg/thesis_hints.pdf}

Für die Anfertigung sämtlicher Arbeiten (auch Bachelor- und Masterarbeit) wird ein \textbf{zweiseitiges Layout} bevorzugt. Ein solches Layout liegt auch dieser Vorlage zugrunde. Um ein einseitiges Layout zu verwenden, muss in der Präambel (\textit{DocumentClass}) die Einstellung \textit{twoside} zu \textit{oneside} geändert werden. Weiterhin wird als Bindungsart die \textbf{Ringbindung} bevorzugt, da dadurch Druckkosten gespart werden können und die Arbeit wesentlich besser zu lesen ist.

Für die Bearbeitung der \LaTeX-Vorlage wird die Verwendung eines \textbf{\LaTeX-Editors} empfohlen. Es gibt eine große Auswahl frei verfügbarer Editoren; weitergehende Informationen sind im Internet zu finden (z.B. \url{https://tex.stackexchange.com/questions/339/latex-editors-ides}). Weiterhin sind Grundkenntnisse in \LaTeX{} erforderlich, da die Vorlage nicht selbsterklärend ist. Eine ganze Reihe verschiedener \LaTeX-Tutorials sind im Internet zu finden.

