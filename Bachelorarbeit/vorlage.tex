%======================================================================================================
% minimale Vorlage für wissenschaftliche Arbeiten Johannes Pöschmann
%======================================================================================================

%======================================================================================================
% DocumentClass
%======================================================================================================
\documentclass[
%	draft,			% Entwurfsmodus: Bilder als Rahmen,
	11pt,			% KOMA default
	a4paper,		% DIN A4
	twoside,		% Zweiseitig (oneside für einseitiges Layout)
	german,			%
	headsepline,		% Linie unter der Kopfzeile
	cleardoublepage=plain,	% Seitenummer auf leeren Seiten
	openany,		% Kapitelüberschriften auf gerader/ungerader Seite
	BCOR8mm,		% Bindungskorrektur 	
	bibliography=totoc,	% Literaturverzeichnis im Inhaltsverzeichnis
	listof=totoc,		% Abbildungs und Tabellenverzeichnis ins Inhaltsverzeichnis
	appendixprefix,		% Anhang
%	abstracton, 		% Abstract/Zusammenfassung
	automark,		% Kapitelnr. und -überschrift in Kopfzeile
]{scrreprt}			% siehe <http://www.komascript.de>

%======================================================================================================
%	Packages
%======================================================================================================
\usepackage{morewrites} % mehr \writes
\usepackage{selinput}		% Eingabecodierung automatisch ermitteln …
\SelectInputMappings{		% … siehe <http://ctan.org/pkg/selinput>
  adieresis={ä},
  germandbls={ß},
}
\usepackage{import}
\usepackage{scrhack} % Fehlerbehebung Floats KOMA
\usepackage{amsmath,amsfonts,amssymb,amsthm} % erweiterte Matheumgebung
%\usepackage{mathtools}
\usepackage[ngerman]{babel}
\usepackage{csquotes}
\usepackage[backend=biber, style=authoryear]{biblatex}
\usepackage[ngerman]{translator} % für glossaries 
%\usepackage[hypertexnames=false]{hyperref}
%\usepackage[
%nonumberlist, %keine Seitenzahlen anzeigen
%acronym,      %ein Abkürzungsverzeichnis erstellen
%toc,          %Einträge im Inhaltsverzeichnis
%% automake,
%section     %im Inhaltsverzeichnis auf section-Ebene erscheinen
%]{glossaries}
\usepackage{scrpage2}				% Kopf- und Fußzeilen flexibel gestalten
\usepackage[onehalfspacing]{setspace}		% 1.5-facher Zeilenabstand

% unterdrückt die Ausgabe der Unterkapitel im Anhang
\usepackage{etoolbox}
\appto\appendix{\addtocontents{toc}{\protect\setcounter{tocdepth}{0}}}
% reinstate the correct level for list of tables and figures
\appto\listoffigures{\addtocontents{lof}{\protect\setcounter{tocdepth}{1}}}
\appto\listoftables{\addtocontents{lot}{\protect\setcounter{tocdepth}{1}}}

%======================================================================================================
%	Bilder, Links
%======================================================================================================

\usepackage{float}
\usepackage{graphicx}
\usepackage{subfig}
\usepackage{wrapfig}
\graphicspath{{img/}}	% Angabe der Pfade, wo die Grafiken liegen;
\newcommand{\mtlstscale}{0.45} % Skalierungsfaktor für Matlab-SVGs mit Standardeinstellungen
\pdfsuppresswarningpagegroup=1

%======================================================================================================
%	Sonstiges
%======================================================================================================
\usepackage[hypertexnames=false]{hyperref}
\usepackage[
nonumberlist, %keine Seitenzahlen anzeigen
acronym,      %ein Abkürzungsverzeichnis erstellen
toc,          %Einträge im Inhaltsverzeichnis
% automake,
section     %im Inhaltsverzeichnis auf section-Ebene erscheinen
]{glossaries}
\usepackage{siunitx}
\newcommand{\bs}[1]{\boldsymbol{#1}}

%======================================================================================================
%	Einstellungen
%======================================================================================================
\setcounter{secnumdepth}{3} % 3 - Überschriften bis subsubsection nummerieren
\pagestyle{scrheadings}				% Standart  Kopf- und Fußzeile
\setkomafont{pageheadfoot}{\small\scshape}	% nur Großbuchstaben 
\setlength{\headheight}{1.1\baselineskip}
\addbibresource{literatur.bib}
% eigenes Symbolverzeichnis erstellen
\newglossary[slg]{symbolslist}{syi}{syg}{Symbolverzeichnis}
\newglossary[malg]{mathlist}{mayi}{mayg}{Generelle mathematische Notationen}
\newglossary[lalg]{latlist}{layi}{layg}{Lateinische Buchstaben}
\newglossary[grlg]{grelist}{gryi}{gryg}{Griechische Buchstaben}
\newglossary[idlg]{idzlist}{idyi}{idyg}{Indizes}
% Punkt am ende jeder Beschreibung deaktivieren
\renewcommand*{\glspostdescription}{}
\makeglossaries
\subimport{glossaries/}{glossar}
\subimport{glossaries/}{abkuerzungen}
\subimport{glossaries/}{mathematische_notation}
\subimport{glossaries/}{symbole}
\subimport{glossaries/}{lateinische_buchstaben}
\subimport{glossaries/}{griechische_buchstaben}
\subimport{glossaries/}{indizes}
\glsaddallunused % Auch nicht referenzierte Glossary-Einträge ausgeben

%======================================================================================================
%	Aufbau, Darstellung etc
%======================================================================================================
\begin{document}
%------- Variablen, \hypersetup, etc. -----------------------------------------------------------------

%       Metadaten
%======================================================================
%       $Id$
%       Matthias Kupfer
%======================================================================

\newcommand{\dcsubject}{Arbeitsart}
% z.B. (Diplom/Studien/Haus)arbeit, Praktikumsbericht, Studie, Beleg, 
% (Pro/Haupt/Ober)seminar, Seminar usw.
\newcommand{\dctitle}{Titel der Arbeit}
\newcommand{\dcsubtitle}{~} % Untertitel, falls erforderlich

\newcommand{\dcauthorlastname}{Mustermann}
\newcommand{\dcauthorfirstname}{Max}
\newcommand{\dcauthoremail}{max.mustermann@s2017.tu-chemnitz.de}
\newcommand{\dcdate}{\today}

\newcommand{\dcplace}{Chemnitz} % Ort, kann an der TU meist so bleiben
\newcommand{\dcuni}{Technische Universität \dcplace}
\newcommand{\dcdepart}{Fakultät für Elektrotechnik und Informationstechnik} % Fakultätsangabe
\newcommand{\dcprof}{Professur Prozessautomatisierung} % Angabe der Professur

\newcommand{\dcpruefer}{Prof. Dr.-Ing. Peter Protzel}% Prüfer der Arbeit
\newcommand{\dcadvisor}{}% Betreuer der Arbeit

\newcommand{\dckeywords}{Liste,von,Stichworten,die,als,Schlagworte,geeignet,
sind}

%%======================================================================
% Einstellungen des Hyperref-Paketes
\hypersetup{%    
        pdftitle        = {\dctitle}, %
        pdfsubject      = {\dcsubject, \dcdate}, %
        pdfauthor       = {\dcauthorfirstname~\dcauthorlastname, \dcauthoremail}, %
        pdfkeywords     = {\dckeywords}, %
        pdfcreator      = {pdfTeX with Hyperref and Thumbpdf}, %
        pdfproducer     = {LaTeX, hyperref, thumbpdf}, %
        % weitere PDF-Einstellungen in hyperref.cfg
}
%%======================================================================
 

%------- Deckblatt ------------------------------------------------------------------------------------
%======================================================================
%	Titelseite 
%======================================================================
%	$Date:$
%	$Revision:$
%	Matthias Kupfer
%======================================================================

%%======================================================================
%% Schmutztitel
%%======================================================================
%\extratitle{
%	\usekomafont{sectioning}\mdseries 
%	\begin{center}
%		\Huge \dcsubject\\[1.5ex]
%		\hrule
%		\vspace*{\fill}
%		\includegraphics{TUC_deutsch_einzeile_CMYK}
%	\end{center}
%}

%%======================================================================
%% Titelkopf
%%======================================================================
\titlehead{
	\vspace*{-1.5cm}
	% Schriftfamilie wie alle Überschriften, aber nicht fett
	\usekomafont{disposition}\mdseries 
	\begin{center}
		\raisebox{-1ex}{\includegraphics[scale=1.4]{img/TUC_deutsch_einzeile_CMYK}}\\
		\hrulefill \\[1em]
		{\Large\dcdepart}\\[0.5em] 
		\dcprof
	\end{center}
	\vspace*{1.5cm}
}

%%======================================================================
%% Subjekt
%%======================================================================
\subject{\textbf{\Huge\dcsubject}}


%%======================================================================
%% Titel
%%======================================================================
\title{\Large
	\dctitle
	\\
	\dcsubtitle
}

%%======================================================================
%% Autor des Dokumentes
%%======================================================================
\author{\dcfirstauthorfirstname~\dcfirstauthorlastname,
~\dcsecondauthorfirstname~\dcsecondauthorlastname}
	
%%======================================================================
%% Ort, Datum
%%======================================================================
\date{\dcplace, \dcdate
}

%%======================================================================
%% Publishers
%%======================================================================
\publishers{
	{\parbox{\textwidth}{
		\begin{tabbing}
			{\textbf{Betreuer:}}\quad\=\kill
			{\textbf{Prüfer:}}	\>\dcpruefer\\
			{\textbf{Betreuer:}}	\>\dcadvisor
		\end{tabbing}	
	}}
}

%%======================================================================
%% bibliografische Angaben
%%======================================================================
\lowertitleback{
\textbf{\dcfirstauthorlastname, \dcfirstauthorfirstname; 
\dcsecondauthorlastname, \dcsecondauthorfirstname}\\
\dctitle\\
\dcsubject,~\dcdepart\\
\dcuni,~\ifcase\month\or
  Januar\or Februar\or März\or April\or Mai\or Juni\or
    Juli\or August\or September\or Oktober\or November\or Dezember\fi
    ~\number\year
}

%%======================================================================
%% maketitle
%%======================================================================

\maketitle 

%------- Kurzfassung / Abstract -----------------------------------------------------------------------
%\def\abstractname{Abstract} 	% auskommentieren für Titel "Zusammenfassung" 	
%\begin{abstract}
%  Hier Inhalt der Arbeit kurz beschreiben. Nicht unbedingt nötig ...
%\end{abstract}

%------- Inhaltsverzeichnis ---------------------------------------------------------------------------
\pagenumbering{roman}
\pdfbookmark{Inhaltsverzeichnis}{Inhaltsverzeichnis}
\tableofcontents

%------- Abbildungsverzeichnis ------------------------------------------------------------------------
\markboth{Abbildungsverzeichnis}{Abbildungsverzeichnis}
\listoffigures

%------- Tabellenverzeichnis --------------------------------------------------------------------------
\markboth{Tabellenverzeichnis}{Tabellenverzeichnis}
\listoftables

%------- Abkürzungen, Symbole --------------------------------------------------------------------
\deftranslation[to=German]{Acronyms}{Abkürzungsverzeichnis}
\chapter{Glossar, Abkürzungs- und Symbolverzeichnis}
\printglossary[style=altlist,title=Glossar]
\printglossary[type=\acronymtype,style=long] % Abkürzung und Langform
\printglossary[type=symbolslist,style=long]
\printglossary[type=mathlist,style=long]
\printglossary[type=latlist,style=long]
\printglossary[type=grelist,style=long]
\printglossary[type=idzlist,style=long]


%------- Fließtext, Kapitel etc -----------------------------------------------------------------------
%\newpage
\cleardoublepage
\pagenumbering{arabic}		% arabische Zahlen für Fließtext
\subimport{chapters/}{chapters}
%\subimport*{chapters/einleitung/}{einleitung}
%\subimport*{chapters/fahrzeug_szenario/}{fahrzeug_szenario}
%\subimport*{chapters/grundlagen/}{grundlagen}
%\subimport*{chapters/software_struktur/}{software_struktur}
%\subimport*{chapters/bildvorverarbeitung/}{bildvorverarbeitung}
%\subimport*{chapters/fahrspurerkennung/}{fahrspurerkennung}
%\subimport*{chapters/regelung/}{regelung}

%------- Literaturverzeichnis -------------------------------------------------------------------------
\newpage
\manualmark
\markboth{Literaturverzeichnis}{Literaturverzeichnis}
\printbibliography

%------- Anhang ----------------------------------------------------------------------------------------
\newpage
\appendix
\addcontentsline{toc}{chapter}{Anhang} % Zeigt "Anhang'' in Inhaltsverzeichnis
\chapter{Daten-CD}	
\section{Digitale Version der Arbeit (PDF-Format)}
\section{Quellcode}
Beispiel Anhang. Hier die Funktionsweise des Quellcodes etc erklären.
\chapter{Ein weiterer Anhang}

%------- Selbstständigkeitserklärung --------------------------------------------------------------------
\chapter*{Selbstständigkeitserklärung}
Hiermit erkläre ich, dass ich die vorliegende Arbeit
selbstständig angefertigt, nicht anderweitig zu Prüfungszwecken vorgelegt und
keine anderen als die angegebenen Hilfsmittel verwendet habe. Sämtliche 
wissentlich verwendete Textausschnitte, Zitate oder Inhalte anderer Verfasser 
wurden ausdrücklich als solche gekennzeichnet.\bigskip\\
\dcplace, \dcdate
\bigskip\bigskip
\flushleft
\newlength\usnl
\settowidth{\usnl}{-\dcfirstauthorfirstname~\dcfirstauthorlastname, \dcsecondauthorfirstname~\dcsecondauthorlastname-}
\begin{tabular}{p{\usnl}}\hline
\centering\footnotesize \dcfirstauthorfirstname~\dcfirstauthorlastname, \dcsecondauthorfirstname~\dcsecondauthorlastname
\end{tabular}

\end{document}

