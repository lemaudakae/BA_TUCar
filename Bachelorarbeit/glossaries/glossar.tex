\newglossaryentry{glos:inlier}{
name=Inlier,
plural=Inliern,
description={Punkte mit einem Abstand kleiner als ein festgelegter Schwellwert von einem gewählten Modell}
}
\newglossaryentry{glos:outlier}{
name=Outlier,
plural=Outliern,
description={Punkte mit einem Abstand größer als ein festgelegter Schwellwert von einem gewählten Modell, d.h. alle Punkte die nicht als \gls{glos:inlier} zählen}
}
\newglossaryentry{glos:scanline}{
name=Scanline,
plural=Scanlines,
description={siehe \ref{ssec:fahrspurerkennung:kalman:messung}}
}
\newglossaryentry{glos:leastsquaresfit}{
name=Least-Squares-Fit,
plural=Least-Squares-Fits,
description={Kurvenanpassung mittels der Methode der kleinsten Quadrate}
}
\newglossaryentry{glos:purepursuit}{
name=Pure-Pursuit,
description={siehe REFERENZ}
}
\newglossaryentry{glos:lookahead-distance}{
name=Lookahead-Distance,
description={Entfernung des Zielpunktes der \gls{glos:purepursuit}-Regelung vom Koordinatenursprung des Roboter-\gls{acr:ks} \gls{lat:RoboterKOS} siehe REFERENZ}
}
\newglossaryentry{glos:node}{
name=Node,
plural=Nodes,
description={\glqq Knoten\grqq Programm, welches via \gls{acr:ros} mit anderen Nodes kommuniziert.}
}
\newglossaryentry{glos:message}{
name=Message,
plural=Messages,
description={ \glqq Nachricht\grqq \gls{acr:ros}-Datentyp welcher genutzt wird, wenn ein \gls{glos:topic} abboniert/veröffentlicht wird.}
}
\newglossaryentry{glos:topic}{
name=Topic,
plural=Topics,
description={ \glqq Thema\grqq \glspl{glos:message} können in ein Topic veröffentlicht werden. Ein Topic kann abboniert werden um \glspl{glos:message} zu empfangen.}
}
\newglossaryentry{glos:subscriber}{
name=Subscriber,
plural=Subscriber,
description={ \glqq Abbonent\grqq eines \gls{glos:topic}s}
}
\newglossaryentry{glos:callback}{
name=Callback,
plural=Callbacks,
description={ \glqq Rückruffunktion\grqq}
}

