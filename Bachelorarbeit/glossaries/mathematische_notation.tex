%\newcommand{\scl}[1]{\ensuremath{\expandafter\MakeLowercase\expandafter{#1}}}
\newcommand{\scl}[1]{\ensuremath{#1}}
\newglossaryentry{math:scalar}{
	name=\scl{x},
	description={Kleinbuchstaben bezeichnen einen Skalar},
	sort=mathscalar, type=mathlist
}
%\newcommand{\pnt}[1]{\ensuremath{\boldsymbol{\expandafter\MakeLowercase\expandafter{#1}}}}
\newcommand{\pnt}[1]{\ensuremath{\boldsymbol{#1}}}
%\newglossaryentry{math:point}{
%	name=\pnt{x},
%	description={Fettgedruckte Kleinbuchstaben bezeichnen einen Punkt},
%	sort=mathpoint, type=mathlist
%}
%\newcommand{\vct}[1]{\ensuremath{\boldsymbol{\expandafter\MakeLowercase\expandafter{#1}}}}
\newcommand{\vct}[1]{\ensuremath{\boldsymbol{#1}}}
%\newglossaryentry{math:vector}{
%name=\vct{x},
%description={Fettgedruckte Kleinbuchstaben bezeichnen einen Vektor},
%sort=mathvector, type=mathlist
%}
\newglossaryentry{math:vectorandpoint}{
name=\vct{x},
description={Fettgedruckte Kleinbuchstaben bezeichnen einen Punkt oder Vektor},
sort=mathvectorandpoint, type=mathlist
}
%\newcommand{\mtx}[1]{\ensuremath{\boldsymbol{\expandafter\MakeUppercase\expandafter{#1}}}}
\newcommand{\mtx}[1]{\ensuremath{\boldsymbol{#1}}}
\newglossaryentry{math:matrix}{
name=\mtx{X},
description={Fettgedruckte Großbuchstaben bezeichnen eine Matrix},
sort=mathmatrix, type=mathlist
}
%\DeclareMathOperator{\fnfop}{f}
\newcommand{\fnfop}{\ensuremath{\mathrm{f}}}
\newcommand{\fnf}[1]{\ensuremath{\fnfop{(#1)}}}
\newglossaryentry{math:functionf}{
	name=\fnf{\scl{x}},
	description={allgemeine Funktion in Abhängigkeit der Variable \scl{x}},
	sort=mathfnf, type=mathlist
}
\newcommand{\dt}[1]{\ensuremath{\dot{#1}}}
\newglossaryentry{math:dt}{
name=\dt{\fnfop},
description={Ableitung der Funktion \ensuremath{f} nach der Zeit},
sort=mathdt, type=mathlist
}
\newcommand{\prd}[1]{\ensuremath{\hat{#1}}}
\newglossaryentry{math:predicted}{
name=\prd{x},
description={Ein Circumflex über einer Variablen symbolisiert eine prädizierte Größe},
sort=mathpredicted, type=mathlist
}
\newcommand{\crd}[1]{\ensuremath{\tilde{#1}}}
\newglossaryentry{math:corrected}{
name=\crd{x},
description={Eine Tilde über einer Variablen symbolisiert eine korrigierte Größe},
sort=mathcorrected, type=mathlist
}
\newcommand{\trp}[1]{\ensuremath{{#1}^T}}
\newglossaryentry{math:trans}{
name=\trp{\mtx{X}},
description={Transponierte der Matrix \mtx{X}},
sort=mathtrans, type=mathlist
}
\newcommand{\ivt}[1]{\ensuremath{{#1}^{-1}}}
\newglossaryentry{math:inv}{
name=\ivt{\mtx{X}},
description={Inverse der Matrix \mtx{X}},
sort=mathinv, type=mathlist
}
\newcommand{\der}[1]{\ensuremath{#1^{\prime}}}
\newcommand{\derII}[1]{\ensuremath{#1^{\prime\prime}}}
\newcommand{\derIII}[1]{\ensuremath{#1^{\prime\prime\prime}}}
\newglossaryentry{math:derivative}{
name=\der{\fnfop},
description={Ableitung der Funktion \fnfop},
sort=mathderivative, type=mathlist
}
\DeclareMathOperator{\maxop}{max}
\newcommand{\mxm}[1]{\ensuremath{\maxop(#1)}}
\newglossaryentry{math:max}{
name=\mxm{\vct{x}},
description={Maxima von \vct{x}},
sort=mathmax, type=mathlist
}
\newcommand{\mean}[1]{\ensuremath{\bar{#1}}}
\newglossaryentry{math:mean}{
name=\mean{x},
description={Mittelwert der Variable \(x\)},
sort=mathmean, type=mathlist
}
\newcommand{\nrm}[1]{\ensuremath{\|#1\|}}
\newglossaryentry{math:norm}{
name=\nrm{\vct{x}},
description={Euklidische Norm des Vektors \({\vct{x}=\begin{pmatrix} x_x \\ x_y \end{pmatrix}}\): \(\nrm{\vct{x}} = \sqrt[2]{x_x^2 + x_y^2}\)},
sort=mathnorm, type=mathlist
}
\newcommand{\btr}[1]{\ensuremath{|#1|}}
\newglossaryentry{math:btr}{
	name=\btr{\scl{x}},
	description={Betrag von \scl{x}},
	sort=mathbtr, type=mathlist
}
\newcommand{\derat}[2]{\ensuremath{{#1}^{\prime} |_{#2}}}
\newglossaryentry{math:derivativeat}{
	name=\derat{\fnfop}{\gls{x}},
	description={Ableitung der Funktion \( \fnfop \) an der Stelle \gls{x}},
	sort=mathderivativeat, type=mathlist
}
\DeclareMathOperator{\atantwoop}{atan2}
\newcommand{\atantwo}[1]{\ensuremath{\atantwoop(#1)}}
\newglossaryentry{math:atan2}{
	name=\atantwo{\gls{y},\gls{x}},
	description={erweiterte Umkehrfunktion des Tangens \(\tan(\alpha)=\frac{\gls{y}}{\gls{x}} \rightarrow \atantwo{\gls{y},\gls{x}} = \alpha\)   mit Wertebereich \(\alpha = 0 \dots 360^{\circ}\)},
	sort=mathatantwo, type=mathlist
}