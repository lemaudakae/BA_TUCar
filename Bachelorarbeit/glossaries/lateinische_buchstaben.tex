% Methodik der Sortierung: 
% sort = lat_{1s - Skalar; 2p - Punkt; 3v - Vektor; 4m - Matrix; 5k-Koordinatensystem}_{Name/Symbol_selbst}

%% ----------- Leopolds Einträge: -------------- %%
% Kalman-Filter
%% Zustandsraum
%%% Vektoren
\newglossaryentry{lat:statevector}{
name=\vct{x},
description={Zustandsvektor},
sort=lat_3v_x, type=latlist
}
\newglossaryentry{lat:inputvector}{
name=\vct{u},
description={Eingangsvariablenvektor},
sort=lat_3v_u, type=latlist
}
\newglossaryentry{lat:outputvector}{
name=\vct{y},
description={Ausgangsvariablenvektor},
sort=lat_3v_y, type=latlist
}
\newglossaryentry{lat:processnoisevector}{
name=\vct{z},
description={Systemrausch-/Prozessrauschvariablenvektor},
sort=lat_3v_z, type=latlist
}
\newglossaryentry{lat:measnoisevector}{
name=\vct{v},
description={Messrauschvariablenvektor},
sort=lat_3v_v, type=latlist
}
%%% Matritzen
\newglossaryentry{lat:systemmatrix}{
name=\mtx{A},
description={Systemmatrix},
plural={Systemmatritzen},
sort=lat_4m_a, type=latlist
}
\newglossaryentry{lat:inputmatrix}{
name=\mtx{B},
description={Eingangsmatrix},
plural={Eingangsmatritzen},
sort=lat_4m_b, type=latlist
}
\newglossaryentry{lat:outputmatrix}{
name=\mtx{C},
description={Ausgangsmatrix/Messmatrix},
sort=lat_4m_c, type=latlist
}
\newglossaryentry{lat:transmissionmatrix}{
name=\mtx{D},
description={Durchgangsmatrix},
sort=lat_4m_d, type=latlist
}
\newglossaryentry{lat:processnoisematrix}{
name=\mtx{G},
description={Prozess-/Systemrauschmatrix},
sort=lat_4m_g, type=latlist
}
%%%% Beobachtbarkeit
\newglossaryentry{lat:observabilitymatrix}{
name=\mtx{S_B},
description={Beobachtbarkeitsmatrix},
sort=lat_4m_s_b, type=latlist
}

%% Filtergleichungen
%%% Korrektur
\newglossaryentry{lat:innovation}{
name=\ensuremath{\Delta\mtx{y}},
description={Innovation},
sort=lat_3v_y_delta, type=latlist
}
\newglossaryentry{lat:kalmangain}{
name=\mtx{K},
description={Kalman-Verstärkungs-Matrix},
sort=lat_4m_k, type=latlist
}
\newglossaryentry{lat:processnoisevariancematrix}{
name=\mtx{Q},
description={Kovarianzmatrix des System-/Prozessrauschens},
sort=lat_4m_q, type=latlist
}
\newglossaryentry{lat:measnoisevariancematrix}{
name=\mtx{R},
description={Kovarianzmatrix des Messrauschens},
sort=lat_4m_r, type=latlist
}
\newglossaryentry{lat:covariancematrix}{
name=\mtx{P},
description={Kovarianzmatrix},
sort=lat_4m_p, type=latlist
}
%% Fahrspurverfolgung
%%% Polynomkoeffizienten
\newglossaryentry{lat:cfpa}{
	name=\scl{a_{fp}},
	description={Koeffizient 0. Ordnung des zur Fahrspurmodellierung genutzten Polynoms},
	sort=lat_1s_a_f_p, type=latlist
}
\newglossaryentry{lat:cfpb}{
	name=\scl{b_{fp}},
	description={Koeffizient 1. Ordnung des zur Fahrspurmodellierung genutzten Polynoms},
	sort=lat_1s_b_f_p, type=latlist
}
\newglossaryentry{lat:cfpc}{
	name=\scl{c_{fp}},
	description={Koeffizient 2. Ordnung des zur Fahrspurmodellierung genutzten Polynoms},
	sort=lat_1s_c_f_p, type=latlist
}
\newglossaryentry{lat:cfpd}{
	name=\scl{d_{fp}},
	description={Koeffizient 3. Ordnung des zur Fahrspurmodellierung genutzten Polynoms},
	sort=lat_1s_d_f_p, type=latlist
}
%%% Zustandsvektor
\newglossaryentry{lat:lateraloffset}{
name=\scl{y_0},
description={Lateraler Versatz: Abstand der Fahrspurmitte vom Koordinatenursprung des Fahrspurkoordinatensystems???},
sort=lat_1s_y_0, type=latlist
}
\newglossaryentry{lat:curvature}{
name=\scl{c_0},
description={Fahrspurkrümmung: Änderungsrate des Gierwinkels},
sort=lat_1s_c_0, type=latlist
}
\newglossaryentry{lat:curvaturechange}{
name=\scl{c_1},
description={Änderungsrate der Fahrspurkrümmung},
sort=lat_1s_c_1, type=latlist
}
%%%zurückgelegter Weg
\newglossaryentry{lat:distpic}{
	name=\scl{\Delta\gls{x}},
	description={zurückgelegter Weg zwischen zwei Bildaufnahmenpunkten},
	sort=lat_1s_c_1, type=latlist
}

% Allgemeines
\newglossaryentry{lat:time}{
name=\scl{t},
description={Zeit},
sort=lat_1s_t, type=latlist
}
\newglossaryentry{lat:velocity}{
name=\scl{v},
description={Geschwindigkeit},
sort=lat_1s_v, type=latlist
}
\newglossaryentry{lat:iter}{
name=\scl{k},
description={Iteration},
sort=lat_1s_k, type=latlist
}
\newglossaryentry{lat:unitmatrix}{
name=\mtx{I},
description={Einheitsmatrix},
sort=lat_4m_i, type=latlist
}
\newglossaryentry{lat:naturalnumber}{
name=\scl{n},
description={natürliche Zahl},
sort=lat_1s_n, type=latlist
}
\newglossaryentry{lat:periodictime}{
name=\scl{t_s},
description={Periodendauer der Abtastung},
sort=lat_1s_t_s, type=latlist
}
\newglossaryentry{x}{
name=\scl{x},
description={x-Koordinate},
sort=lat_1s_x, type=latlist
}
\newglossaryentry{y}{
name=\scl{y},
description={y-Koordinate},
sort=lat_1s_y, type=latlist
}
\newglossaryentry{z}{
name=\scl{z},
description={z-Koordinate},
sort=lat_1s_z, type=latlist
}
\newglossaryentry{lat:eulnum}{
name=\scl{e},
description={eulersche Zahl \(e \approx 2,718\)},
sort=lat_1s_e, type=latlist
}
%	\newglossaryentry{lat:func}{
%	name=\ensuremath{f},
%	description={Funktion},
%	sort=lat_0_f, type=latlist
%	}

% Kameramodell
\newglossaryentry{lat:ocv}{ %lat:cameravector
name=\vct{v_{\gls{lat:OCKOS}}},
description={Vom optischen Zentrum des Objektivs ausgehender Vektor},
sort=lat_3v_v_o_c, type=latlist
}
\newglossaryentry{lat:osp}{ %lat:camerapoint
name=\pnt{s_{\gls{lat:OSKOS}}},
description={Punkt auf dem Bildsensor},
sort=lat_2p_s_o_s, type=latlist
}
\newglossaryentry{lat:ocp}{
name=\pnt{p_{\gls{lat:OCKOS}}},
description={Punkt in der zur Rektifizierung erstellten Ebene},
sort=lat_2p_p_o_s, type=latlist
}
%% Polynomkoeffizienten
\newglossaryentry{lat:copa}{
name=\scl{a_{op}},
description={Koeffizient 0. Ordnung des bei der Kamerakalibration ermittelten Polynoms},
sort=lat_1s_a_o_p, type=latlist
}
\newglossaryentry{lat:copb}{
name=\scl{b_{op}},
description={Koeffizient 1. Ordnung des bei der Kamerakalibration ermittelten Polynoms},
sort=lat_1s_b_o_p, type=latlist
}
\newglossaryentry{lat:copc}{
name=\scl{c_{op}},
description={Koeffizient 2. Ordnung des bei der Kamerakalibration ermittelten Polynoms},
sort=lat_1s_c_o_p, type=latlist
}
\newglossaryentry{lat:copd}{
name=\scl{d_{op}},
description={Koeffizient 3. Ordnung des bei der Kamerakalibration ermittelten Polynoms},
sort=lat_1s_d_o_p, type=latlist
}
\newglossaryentry{lat:cope}{
name=\scl{e_{op}},
description={Koeffizient 4. Ordnung des bei der Kamerakalibration ermittelten Polynoms},
sort=lat_1s_e_o_p, type=latlist
}
%% affine Transformation
%%% Koeffizienten
\newglossaryentry{lat:coac}{
name=\scl{c_{oa}},
description={Koeffizient der bei der Kamerakalibration ermittelten affinen Transformation},
sort=lat_1s_c_o_a, type=latlist
}
\newglossaryentry{lat:coad}{
name=\scl{d_{oa}},
description={Koeffizient der bei der Kamerakalibration ermittelten affinen Transformation},
sort=lat_1s_d_o_a, type=latlist
}
\newglossaryentry{lat:coae}{
name=\scl{e_{oa}},
description={Koeffizient der bei der Kamerakalibration ermittelten affinen Transformation},
sort=lat_1s_e_o_a, type=latlist
}
%%% Verschiebungsvektor Bildmittelpunkt
\newglossaryentry{lat:voa}{
name=\vct{c_{\gls{lat:OSKOSR}\_\gls{lat:OSKOS}}},
description={Verschiebungsvektor des Bildmittelpunktes (Koordinatenursprungs) von \gls{lat:OSKOS} zu \gls{lat:OSKOSR}},
sort=lat_3v_c_o_a, type=latlist
}
%Riverflow
%%Randlinie
\newglossaryentry{lat:vvrrp}{
	name=\vct{v_{rrp\gls{lat:BildKOS}}},
	description={Verschiebungsvektor zwischen 2 auf der Randline gefundenen Punkten},
	sort=lat_3v_v_v_r_r_p, type=latlist
}
\newglossaryentry{lat:vvrrpx}{
	name=\scl{\gls{x}_{\gls{lat:vvrrp}}},
	description={x-Koordinate des Verschiebungsvektors zwischen 2 auf der Randline gefundenen Punkten},
	sort=lat_1s_v_v_r_r_p_x, type=latlist
}
\newglossaryentry{lat:vvrrpy}{
	name=\scl{\gls{y}_{\gls{lat:vvrrp}}},
	description={y-Koordinate des Verschiebungsvektors zwischen 2 auf der Randline gefundenen Punkten},
	sort=lat_1s_v_v_r_r_p_y, type=latlist
}
\newglossaryentry{lat:prr}{
	name=\pnt{p_{rr\gls{lat:BildKOS}}},
	description={auf der Randline gefundener Punkt},
	sort=lat_2p_p_r_r, type=latlist
}
\newglossaryentry{lat:mrrs}{
	name=\pnt{m_{rrs\gls{lat:BildKOS}}},
	description={Mittelpunkt einer Scanline},
	sort=lat_2p_m_r_r_s, type=latlist
}


% ----------- Daniels Einträge: ------------------ %%
% Transformationsmatrizen
\newglossaryentry{lat:ks_punkt}{
name=\pnt{q},
description={Punkt allgemein in einem Koordinatensystem},
sort=lat_2p_q, type=latlist
}
\newglossaryentry{lat:Translationsvektor}{
name=\vct{t},
description={Translationsvektor},
sort=lat_3v_t, type=latlist
}
\newglossaryentry{lat:Rotationsmatrix}{
name=\mtx{R_z},
description={Rotationsmatrix (für die Rotation um die z-Achse)},
sort=lat_4m_r_z, type=latlist
}
\newglossaryentry{lat:Transformationsmatrix}{
name=\mtx{T},
description={homogene Transformationsmatrix},
sort=lat_4m_t, type=latlist
}
% Koordinatensysteme
\newglossaryentry{lat:WeltKOS}{
name=\ensuremath{\mathcal{W}},
description={Weltkoordinatensystem},
sort=lat_5k_w, type=latlist
}
\newglossaryentry{lat:RoboterKOS}{
name=\ensuremath{\mathcal{R}},
description={Roboterkoordinatensystem},
sort=lat_5k_r, type=latlist
}
\newglossaryentry{lat:BildKOS}{
name=\ensuremath{\mathcal{B}},
description={Bildkoordinatensystem},
sort=lat_5k_b, type=latlist
}
\newglossaryentry{lat:LinienKOS}{
name=\ensuremath{\mathcal{L}},
description={Linienkoordinatensystem},
sort=lat_5k_l, type=latlist
}
\newglossaryentry{lat:AllgKOS}{
name=\ensuremath{\mathcal{K}},
description={Bezeichnung für ein allgemeines Koordinatensystem},
sort=lat_5k_k, type=latlist
}
\newglossaryentry{lat:OCKOS}{
name=\ensuremath{\mathcal{OC}},
description={Bezeichnung für das Kamera-\gls{acr:ks} der Kamerakalibrations-Toolbox},
sort=lat_5k_oc, type=latlist
}
\newglossaryentry{lat:OSKOS}{
name=\ensuremath{\mathcal{OS}},
description={Bezeichnung für das ideale Bildsensor-\gls{acr:ks} der Kamerakalibrations-Toolbox},
sort=lat_5k_os, type=latlist
}
\newglossaryentry{lat:OSKOSR}{
name=\ensuremath{\mathcal{OSR}},
description={Bezeichnung für das reale Bildsensor-\gls{acr:ks} der Kamerakalibrations-Toolbox},
sort=lat_5k_osr, type=latlist
}

% Fahrzeug Abmessung Radstand
\newglossaryentry{lat:radstand}{
name=\scl{l},
description={Radstand (mittlerer Abstand zwischen Vorder-und Hinterachse eines Fahrzeugs)},
sort=lat_1s_l, type=latlist
}
% KS-Komponenten
\newglossaryentry{lat:x_Komp_t}{
name=\scl{a_t},
description={x-Komponente vom Translationsvektor},
sort=lat_1s_a_t, type=latlist
}
\newglossaryentry{lat:y_Komp_t}{
name=\scl{b_t},
description={y-Komponente vom Translationsvektor},
sort=lat_1s_b_t, type=latlist
}
\newglossaryentry{lat:z_Komp_t}{
name=\scl{c_t},
description={z-Komponente vom Translationsvektor},
sort=lat_1s_c_t, type=latlist
}
% Bildmatrizen
\newglossaryentry{lat:BildmatrixOriginal}{
name=\mtx{M},
description={Matrix des originalen Bildes},
sort=lat_4m_m, type=latlist
}
\newglossaryentry{lat:BildmatrixGefiltert}{
name=\ensuremath{\mtx{M}^{\ast}},
description={Matrix des gefilterten Bildes},
sort=lat_4m_m_stern, type=latlist
}
\newglossaryentry{lat:Filterkernmatrix}{
name=\mtx{F},
description={Matrix des Filterkerns eines Faltungsfilters},
sort=lat_4m_f, type=latlist
}
\newglossaryentry{lat:FilterkernZeilen}{
	name=\scl{q},
	description={Zeilen der Faltungsfilterkernmatrix},
	sort=lat_1s_q, type=latlist
}
\newglossaryentry{lat:FilterkernSpalten}{
	name=\scl{r},
	description={Spalten der Faltungsfilterkernmatrix},
	sort=lat_1s_r, type=latlist
}
\newglossaryentry{lat:FilterkernMitteX}{
	name=\scl{x_m},
	description={x-Komponente der Mittelpunktes des Faltungsfilterkerns},
	sort=lat_1s_x_m, type=latlist
}
\newglossaryentry{lat:FilterkernMitteY}{
	name=\scl{y_m},
	description={y-Komponente der Mittelpunktes des Faltungsfilterkerns},
	sort=lat_1s_y_m, type=latlist
}
\newglossaryentry{lat:LaufvariableI}{
	name=\scl{i},
	description={Laufvariable},
	sort=lat_1s_i, type=latlist
}
\newglossaryentry{lat:LaufvariableJ}{
	name=\scl{j},
	description={Laufvariable},
	sort=lat_1s_j, type=latlist
}
\newglossaryentry{lat:normalverteilung}{
	name=\scl{h},
	description={Dichtefunktion der Gauß- oder Normalverteilung},
	sort=lat_1s_h, type=latlist
}
\newglossaryentry{lat:houghAchsenabschnitt}{
	name=\scl{a_h},
	description={Achsenabschnittsparameter im Datenraum bei der Houghtransformation},
	sort=lat_1s_a_h, type=latlist
}
\newglossaryentry{lat:houghAnstieg}{
	name=\scl{b_h},
	description={Anstiegsparameter im Datenraum bei der Houghtransformation},
	sort=lat_1s_b_h, type=latlist
}
