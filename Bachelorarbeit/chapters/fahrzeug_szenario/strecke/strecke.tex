\section{Strecke}

\begin{figure}[H] % [htb]
  \centering
  \includegraphics[width=1\textwidth]{strassenszenario.pdf}
  \caption{Die Teststrecke mit Kreuzungen, Einbahnstraßen und Parkplätzen}
  \label{fig:strassenszenario}
\end{figure}

Das in Abbildung~\ref{fig:strassenszenario} dargestellte Straßenverkehrsszenario ist unter mehreren Maßgaben entworfen und auf zwei 4x2 Meter große \gls{glos:pvc}-Werbeplanen gedruckt worden. Diese Planen sind relativ robust und können jederzeit aufgerollt und wie ein Puzzle zusammengelegt werden. Die Breite einer Fahrbahn haben wir anhand des Autos auf 20 cm festgelegt. Damit das Modellfahrzeug ordnungsgemäß durch das Verkehrssystem navigiert werden kann, durfte der Radius der Kurven nicht zu klein sein, da das Auto einen maximalen Lenkwinkel von circa 30\(^\circ\) besitzt. Die Radien der Kurven sind etwa doppelt so groß wie der kleinste mit dem Fahrzeug fahrbare Radius, damit in der Kurve noch geregelt und korrigiert werden kann. Weitläufiger konnten die Kurven leider nicht gemacht werden, da das Platzangebot im Labor nicht ausreichend groß gewesen ist.
Die Zeichnung ist unter dem Gesichtspunkt der Weiterverwendung entstanden. Es ist angedacht, perspektivisch die Fahrbahnumgebung auch für ein Praktikum zu nutzen, in dem viele Aufgabenstellungen denkbar sind. Auch wenn bei weitem nicht alles Ziel unserer Arbeit ist, sollte für das Fahrzeug auf dieser Strecke neben einer normalen Fahrspurverfolgung auch Überholen, Abbiegen, Vorfahrt gewähren, Erkennen einer Einbahnstraße und Einparken in Parktaschen verschiedener Ausführung implementiert werden können. Die Parkplätze sind ebenfalls auf speziell dieses Auto angepasst. Gegenüber eines einfachen Rundkurses ist hier die Herausforderung, eine robuste Fahrspurerkennung zu entwickeln, durch Zusätze wie Querstraßen, Kreuzungen und Parkplätze etwas größer. Da der Platz sehr begrenzt ist, wurde darauf geachtet, möglichst viel der zur Verfügung gestellten Fläche mit Straße zu versehen. Zur optischen Hervorhebung der Fahrbahn und der für das Auge visuellen Ansprechbarkeit ist der Untergrund mit einem hellen Grün bedruckt.
