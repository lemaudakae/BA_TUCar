\section{Hardware}
\subsection{Fahrzeug}
Das in dieser Arbeit genutzte Modellfahrzeug wurde uns freundlicherweise von der Professur zur Verfügung gestellt. Es handelt sich um ein autoähnliches Allrad-Chassis im Maßstab 1:18, an welchem für diese Arbeit notwendige Umbauten ausgeführt wurden.
\paragraph{Motor}
Der originale Elektromotor wurde gegen ein Modell mit Encoder ersetzt, da Odometrieinformationen für die implementierte Regelung unerlässlich sind. Zusätzlich ist dem Motor ein Getriebe nachgeschaltet, welches ein Fahren mit langsameren Geschwindigkeiten ermöglicht. Tests mit wenig optimierten Algorithmen werden somit signifikant erleichtert.
\paragraph{Kamera}
Neben der Odometrie stellt eine in Vogelperspektive angebrachte Kamera mit Fisheye-Objektiv die einzige Sensorik des Fahrzeugs dar.
\paragraph{Rechentechnik}
Zur Ansteuerung der Kamera und zum Entgegennehmen der Fahrbefehle ist ein Raspberry-PI Einplatinenrechner montiert, dessen WLAN-Schnittstelle zum Informationsaustausch mit dem Fahrzeug dient.
