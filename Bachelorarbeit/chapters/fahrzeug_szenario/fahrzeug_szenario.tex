\chapter{Modellfahrzeug \& Testzenario \dcsecondauthorshort}
\label{cha:fahrzeug_szenario}

Die präziseste Forschung zum Themengebiet autonomes Fahren kann mit echten PKW im realen Straßenverkehr oder auf einer daran angelehnten Teststrecke stattfinden. Da dies jedoch weder technisch, finanziell oder rechtlich möglich ist und die im Bachelorstudium erworbenen Fähigkeiten oftmals übersteigt, muss ein geeignetes Modell geschaffen werden. Neben der notwendigen Reduktion der Komplexität wurden die signifikantesten Rahmenbedingungen hierbei durch:
\begin{itemize}
\item die Größe des Labors, in dem die Testumgebung aufgebaut werden sollte
\item die Menge der käuflichen Fahrplattformen mit Ackermannlenkung
\end{itemize}
vorgegeben. 

Weiterhin wurde bewusst auf die Nutzung \glqq einfacher\grqq\ Sensoren wie Infrarot-Abstandsdetektoren, Anschlagsschalter oder Ultraschall-Entfernungsmesser verzichtet, da an der Professur schon eine Fahrplattform mit diesen Sensoren existiert und sie wenig Mehrwert zu den Informationen eines Kamerabildes bieten. Die Umbauten am gewählten Modellfahrzeug \glqq TUCar\grqq\ sowie die Überlegungen zum Entwurf der Strecke sollen in den anschließenden zwei Abschnitten kurz erläutert werden.

\subimport{hardware/}{hardware}
\subimport{strecke/}{strecke}
