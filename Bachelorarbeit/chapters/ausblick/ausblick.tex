\chapter{Ausblick auf folgende Arbeiten \dcfirstauthorshort}
\label{cha:ausblick}

Nachdem innerhalb dieser Gruppenarbeit eine stabile Fahrspurverfolgung mit Kartenaufbau in die Praxis umgesetzt wurde, liegen weiterhin viele Ideen zur Verbesserung, Ergänzung und zum Ausbau der Komplexität der Fähigkeiten des Autos vor. Da dieses Projekt der Anstoß für einen Praktikumsversuch sein soll, ist die Liste der Möglichkeiten lang. Denkbare und zukünftig erweiterbare Implementierungen für das \gls{glos:tucar} wären z.B.:

\begin{itemize}
\item Anhalten an einer Kreuzung gleichrangiger Straßen
	\begin{itemize}
	\item setzt eine Kreuzungserkennung voraus
	\item Vorfahrt gewähren \glqq rechts vor links\grqq
	\item Erkennen eines zweiten Fahrzeugs (z.B. mittels QR-Code)
	\end{itemize}
\item Abbiegen an Kreuzungen und Abzweigen
\item Überholmanöver
\item Hinderniserkennung mit Anhalten und/oder Ausweichen 
\item Einparken
	\begin{itemize}
	\item in Längs - und Querparkplätze
	\item Zwischen Kartons oder andere Autos bei Längsparkplätzen
	\item vorwärts und rückwärts einparken bei Querparkplätzen
	\end{itemize}
\item Variation der Geschwindigkeit in Abhängigkeit von aufgestellten/aufgelegten Geschwindigkeitsbegrenzungen
\end{itemize}
