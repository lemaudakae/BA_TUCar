\section{Hough-Transformation \dcfirstauthorshort}
\label{sec:grundlagen:hough}

Leicht parametrierbare geometrische Objekte in einem Binärbild (z.B. nach einer Kantenerkennung) lassen sich mit dem robusten Verfahren der Hough-Transformation finden. Dessen häufigste Anwendung besteht in der Geradenerkennung, weshalb das Prinzip im Folgenden an diesem einfachen Fall erklärt werden soll. Analog lässt sich die Vorgehensweise auf andere parametrische Figuren übertragen, wie zum Beispiel Kreise.

\subsection{Herleitung}
Um gerade Linien mittels Hough-Transformation in einem Bild finden zu können, ist ein Modell der Gerade, also die mathematische Beschreibung des zu findenden Objektes, notwendig. Alle Punkte (\gls{x},\gls{y}), die auf einer Geraden mit den Parametern \gls{lat:anstieg} und \gls{lat:achsenabschnitt} liegen, lassen sich in Anlehnung an \autocite[S.~481f]{jaehneDigitaleBildverarbeitungMit2005} nach folgender Gleichung beschreiben:

% Formel für allgemeine Geradengleichung
\begin{equation}
\gls{y} = \gls{lat:anstieg} \cdot \gls{x} + \gls{lat:achsenabschnitt}
\label{eq:geradengleichung}
\end{equation}

Hierbei geben \gls{lat:achsenabschnitt} den Achsenabschnitt und \gls{lat:anstieg} den Anstieg der Geraden an. Die Geradengleichung~\eqref{eq:geradengleichung} lässt sich nun so umstellen, dass der Anstieg \gls{lat:anstieg} in Abhängigkeit vom Achsenabschnitt \gls{lat:achsenabschnitt}  dargestellt wird:

% Formel für umgestellte Geradengleichung
\begin{equation}
\gls{lat:anstieg} = -\frac{1}{\gls{x}} \cdot \gls{lat:achsenabschnitt} + \frac{\gls{y}}{\gls{x}}
%\label{eq:geradengleichung_umgest}
\end{equation}

Damit wird ein zweiter sogenannter Modellraum aufgespannt, indem \gls{lat:achsenabschnitt} und \gls{lat:anstieg} zum Definitions- und Wertebereich gesetzt werden und die Punkte zu den Parametern zählen. Der Hough-Algorithmus transformiert jeden Bildpunkt in den Modellraum, welcher dort eine Gerade repräsentiert. Umgekehrt stellt ein Punkt im Modellraum eine Gerade im Datenraum dar. Viele im Bild auf einer gesuchten Gerade liegenden Pixel machen sich dann als häufiger Schnittpunkt im Modellraum bemerkbar. Jener Punkt (\gls{lat:achsenabschnitt},\gls{lat:anstieg}) enthält eine solide Parameterabschätzung der gesuchten Gerade im Bild.

% Bild (Diagramme) zur Hough-Trafo
\begin{figure}[htbp] % [htb]
  \centering  
  \subfloat[Datenraum]{\includegraphics[width=0.48\textwidth]{grundlagen_hough_datenraum.png}}
  \hfill
  \subfloat[Modellraum]{\includegraphics[width=0.48\textwidth]{grundlagen_hough_modellraum.png}}  
  \caption{Houghtransformation von Geraden gemäß \autocite[S.~482]{jaehneDigitaleBildverarbeitungMit2005}: Punkte aus dem Datenraum werden auf den Modellraum abgebildet.}
%  \label{fig:grundlagen_hough}
\end{figure} 

Praktischerweise wird als Modell für die Gerade die hessesche Normalform (Steigungswinkel und Abstand zum Koordinatenursprung als Parameter) eingesetzt, da die Beschreibung in Gleichung~\eqref{eq:geradengleichung} den großen Nachteil hat, dass der Anstiegsparameter \gls{lat:anstieg} bei vertikalen Geraden ins Unendliche läuft. 

\subsection{Vereinfachte Hough-Transformation}
\label{ssec:grundlagen:hough:vereinfachte}
% Hier fehlt der Verweis auf BA_Marcel_Unger
In dieser Arbeit wurde wie in \autocite{alyRealTimeDetection2008} eine sehr vereinfachte Form der Hough-Transformation zur Anwendung gebracht, welche lediglich zur Findung vertikaler Linien dient. Dazu wird ein Histogramm über die Spaltensummen der Bildmatrix gebildet. Nach der Glättung mittels Gauß-Filter geben die Lage der Maxima über einem bestimmten Schwellwert anschließend Auskunft über die Position einer näherungsweise vertikal erkannten Gerade.

% einer alternativen Initialisierung eines Startpunktes zum Einsatz kommt.