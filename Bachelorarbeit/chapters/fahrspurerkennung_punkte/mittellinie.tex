\section{Mittellinie \dcfirstauthorshort}
\label{ssec:fahrspurerkennung:riverflow:mittellinie}
Ähnlich wie im Kapitel~\ref{sec:maskenbau} beschrieben findet dieser Ansatz die zur Mittellinie zugehörigen Elemente, ohne auf richtig definierte \gls{acr:roi} angewiesen zu sein. Ausgangspunkt ist ebenfalls das binarisierte Bild, auf das die MATLAB-Funktion \emph{regionprops} angewendet wird. Diese bereits verfügbare Funktion findet als Alternative zum in Abschnitt~\ref{sec:maskenbau} erwähnten \glqq Ringfilter\grqq{} alle zusammenhängende Pixelgruppen und extrahiert zusätzlich eine Vielzahl derer Eigenschaften, wie Länge, Breite, Orientierung und Mittelpunkt. Da wie in Punkt~\ref{item:riverflow:rule:dashedline} bereits erwähnt die mittlere Fahrbahnmarkierung genau spezifiziert ist, werden so alle potentiellen Mittellinienstriche vorgemerkt. Diese Vorgehensweise funktioniert allerdings nur dann stabil, wenn die Beschaffenheit des binarisierten Bildes gut genug ist, sodass z.B. Striche nicht unterbrochen sind. Trotz der guten Qualität des Bildes kommt es vor, dass falsche Bildteile als Mittellinienelemente erkannt und vorgemerkt werden. Deswegen stellen wir durch eine Verifikation sicher, dass möglichst keine falschen Punkte in die Weltkarte, welche später zur Regelung des Autos genutzt wird, eingetragen werden. Schon an der großen Vier-Seiten-Kreuzung kommt es sonst zum Problem, dass darüber hinaus Mittellinienstriche der kreuzenden Straße erkannt werden. Auch wenn diese zur mittleren Linie gehören, sollen sie hier auf Grund des derzeitigen Regelungskonzepts noch nicht in die Karte aufgenommen werden. 
% Das ist freilich nicht falsch, aber für das aktuelle Regelungskonzept auch nicht zielführend. 