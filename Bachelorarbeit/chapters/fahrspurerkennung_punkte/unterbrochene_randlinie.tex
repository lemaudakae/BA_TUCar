\section{Sonderfall unterbrochene Randlinie \dcsecondauthorshort} 
\label{sssec:fahrspurerkennung:riverflow:randlinie:gestrichelt}
%Da das bis zu diesem Punkt diskutierte Verfahren zur Erkennung der seitlichen Fahrbahnmarkierungen nur zum Verfolgen durchgängiger Linien konzipiert ist, einmündende Straßen jedoch durch eine unterbrochene Markierung gekennzeichnet sind, findet für diesen Sonderfall eine modifizierte Version der in Passage~\ref{ssec:fahrspurerkennung:riverflow:mittellinie} beschriebenen Vorgehensweise Anwendung. Ist Abbruchbedingung \ref{item:fahrspurerkennung:riverflow:randlinie:ende:keinpunkt} des zentralen Algorithmus erreicht, wird ein Suchfenster (siehe Abb.~\ref{fig:riverflow:randlinien:plot_komplett}) festgelegt. Wurden in diesem Bereich durch die Matlab-Funktion \emph{regionprops} Elemente einer kurzen, unterbrochenen Linie gefunden, werden diese als nächste Punkte der seitlichen Fahrbahnmarkierung vermerkt. Die Bildung des Mittelpunkts \pnt{m} des nächsten Suchfensters läuft wie in \eqref{eq:riverflow:solidline:scanlinemidpoint} beschrieben ab. Analog kann bei Fehlen eines Linienelementes im Suchfenster durch Bilden einer Scanline ein Rücksprung zum Fall einer durchgängigen Randlinie vollzogen werden.

Da das in Abschnitt \ref{ssec:fahrspurerkennung:riverflow:randlinie} diskutierte Verfahren zur Erkennung der seitlichen Fahrbahnmarkierungen nur zum Verfolgen durchgängiger Linien konzipiert ist, einmündende Straßen jedoch durch eine unterbrochene Markierung gekennzeichnet sind, findet für diesen Sonderfall eine modifizierte Version der in Passage~\ref{ssec:fahrspurerkennung:riverflow:mittellinie} sowie \ref{ssec:fahrspurerkennung_punkt:verifikation:mittellinie} beschriebenen Vorgehensweise zur Erkennung\&Verifikation der Mittellinie Anwendung. Ist Abbruchbedingung \ref{item:fahrspurerkennung:riverflow:randlinie:ende:keinpunkt} (kein Fahrbahnmarkierungspunkt auf der Scanline gefunden) des Algorithmus zur Detektion der durchgängigen Markierungen erreicht, wird ein Suchfenster (siehe pinke Quadrate in Abb.~\ref{fig:riverflow:randlinien:plot_komplett}) festgelegt. Wurden in diesem Bereich durch die Matlab-Funktion \emph{regionprops} Elemente der Länge und Breite gefunden, welche auf eine unterbrochene Randlinie hindeuten, werden diese als nächste Punkte der seitlichen Fahrbahnmarkierung vermerkt. Die Bildung des Mittelpunkts \pnt{m} des Suchfensters geschieht aufgrund der kurzen Länge eines Randlinienstriches nicht wie in Passage \ref{ssec:fahrspurerkennung_punkt:verifikation:mittellinie} auf Basis der Orientierung dessen, sondern wie in \eqref{eq:riverflow:solidline:scanlinemidpoint} beschrieben durch den Verschiebungsvektor zwischen den letzten zwei gefundenen Punkten der seitlichen Fahrbahnmarkierung. 

Bei Fehlen eines Linienelementes im Suchfenster kann durch Bilden einer Scanline ebenfalls der Rücksprung zum Fall einer durchgängigen Randlinie vollzogen werden.