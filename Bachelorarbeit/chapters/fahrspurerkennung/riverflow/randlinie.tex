\subsection{Randlinie}
\subsubsection{Startpunktgewinnung}
Um den eigentlichen Riverflow-Algorithmus ausführen zu können werden wird die ungefähre Lage der seitlichen Fahrbahnmarkierungen in der Nähe des Fahrzeugs benötigt. Zur Ermittlung dieser gibt es 3 Möglichkeiten:
\begin{enumerate}
\item \label{item:solidline:startpoints:dashedline}
Die Bestimmung durch Orientierung und Position des ersten vor dem Fahzeug gelegenen Mittellinienelementes. Hierzu wird der Mittelpunkt des Elementes entlang der Orientierung desselbigen um die Fahrspurbreite nach links/rechts verschoben. BILD
\item \label{item:solidline:startpoints:hough}
Eine eindimensionale Hough-Transformation des Bildausschnittes direkt vor dem Fahrzeug wie in REFERENZ beschrieben. BILD
\item \label{item:solidline:startpoints:fixed}
Annahme einer festen Position vor dem Fahrzeug/im Kamerabild.
\end{enumerate}

\subsubsection{zentraler Algorithmus}
Durch Nutzung der Eigenschaften \ref{item:riverflow:rule:solidline} und  \ref{item:riverflow:rule:curvature} kann ausgehend von der aktuellen Linienorientierung in einem bestimmten Kegel, dessen Öffnungswinkel durch die maximale Krümmung der Fahrspur definiert ist, der weitere Verlauf der Fahrbahnmarkierung vermutet werden. BILD
Algorithmisch genutzt wird dieses Wissen durch die Verwendung der aus \ref{ssec:fahrspurerkennung:kalman:messung} bekannten \glspl{glos:scanline}.
Zuerst wird ein Verschiebungsvektor \begin{math} \boldsymbol{v_n} \end{math} zwischen vorletztem (\begin{math} \boldsymbol{p_{n-1}} \end{math}) und letztem gefundenen Punkt  (\begin{math} \boldsymbol{p_n} \end{math})  der Fahrbahnmarkierung gebildet \eqref{eq:riverflow:solidline:dispvec}. Ausgehend vom letzten gefundenen Punkt der Linie (\begin{math} \boldsymbol{p_n} \end{math}) wird selbiger nun um den Vektor \begin{math} \boldsymbol{v_n} \end{math} weiterverschoben und bildet den Mittelpunkt  \begin{math} \boldsymbol{m_{n+1}}  \end{math} der nächsten \gls{glos:scanline} \eqref{eq:riverflow:solidline:scanlinemidpoint}.
\begin{equation}
\label{eq:riverflow:solidline:dispvec}
\boldsymbol{v_n} =  \boldsymbol{p_n} - \boldsymbol{p_{n-1}}
\end{equation}
\begin{equation}
\label{eq:riverflow:solidline:scanlinemidpoint}
\boldsymbol{m_{n+1}} =  \boldsymbol{p_n} + \boldsymbol{v_n}
\end{equation}
 Die \gls{glos:scanline} kann nun durch senkrecht zum Verschiebungsvektor \begin{math} \boldsymbol{v_n} \end{math}  durch den Punkt \(\boldsymbol{m_{n+1}}\) konstruiert werden.
Da die Filterantwort des Kantendetekors durch den Algorithmus zur Erkennung der Mittellinie schon für das gesamte Bild vorliegt, kann direkt nach Maxima unter den Pixeln, welche die \gls{glos:scanline} schneidet, gesucht werden

\subsubsection{Sonderfall gestrichelte Randlinien}
\subsubsection{Ende}