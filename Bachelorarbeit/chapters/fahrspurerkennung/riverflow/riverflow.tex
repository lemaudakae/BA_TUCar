\section{Riverflow} \label{sec:fahrspurerkennung:riverflow}
Der dritte und final genutzte Ansatz zur Fahrspurerkennung nutzt im Gegensatz zu den vorherigen Varianten kein durch wenige Parameter (Polynomkoeffizienten etc.) beschreibbares Modell. Dem Objekt Fahrspur werden hingegen bestimmte Eigenschaften zugesprochen, welche zur Idee des Riverflow-Algorithmus \autocite{limRiverFlowLane2012} und der angeschlossenen Punkteverifikation geführt haben.
\begin{enumerate}
\item \label{item:riverflow:rule:solidline}
Ist eine Fahrbahnmarkierung eine Randlinie, so ist dies eine nicht unterbrochene Markierung bestimmter Breite.
\item \label{item:riverflow:rule:dashedline}
Ist eine Fahrbahnmarkierung eine Mittellinie, so besteht diese aus Elementen deren Länge, Breite und Abstand voneinander konstant und bekannt ist.
\item \label{item:riverflow:rule:curvature}
Die Krümmung einer Fahrbahnmarkierung darf einen bestimmten Wert nicht überschreiten.
\item \label{item:riverflow:rule:distance}
Rechte-, Mittel- und Seitenlinie besitzen einen konstanten Abstand (Fahrspurbreite) zueinander.
\end{enumerate} 
\subsection{Mittellinie \dcfirstauthorshort}
\label{ssec:fahrspurerkennung:riverflow:mittellinie}
Ähnlich wie im Kapitel~\ref{sec:maskenbau} beschrieben findet dieser Ansatz die zur Mittellinie zugehörigen Elemente, ohne auf richtig definierte \gls{acr:roi} angewiesen zu sein. Ausgangspunkt ist ebenfalls das binarisierte Bild, auf das die MATLAB-Funktion \emph{regionprops} angewendet wird. Diese bereits verfügbare Funktion findet als Alternative zum in Abschnitt~\ref{sec:maskenbau} erwähnten \glqq Ringfilter\grqq{} alle zusammenhängende Pixelgruppen und extrahiert zusätzlich eine Vielzahl derer Eigenschaften, wie Länge, Breite, Orientierung und Mittelpunkt. Da wie in Punkt~\ref{item:riverflow:rule:dashedline} bereits erwähnt die mittlere Fahrbahnmarkierung genau spezifiziert ist, werden so alle potentiellen Mittellinienstriche vorgemerkt. Diese Vorgehensweise funktioniert allerdings nur dann stabil, wenn die Beschaffenheit des binarisierten Bildes gut genug ist, sodass z.B. Striche nicht unterbrochen sind. Trotz der guten Qualität des Bildes kommt es vor, dass falsche Bildteile als Mittellinienelemente erkannt und vorgemerkt werden. Deswegen stellen wir durch eine Verifikation sicher, dass möglichst keine falschen Punkte in die Weltkarte, welche später zur Regelung des Autos genutzt wird, eingetragen werden. Schon an der großen Vier-Seiten-Kreuzung kommt es sonst zum Problem, dass darüber hinaus Mittellinienstriche der kreuzenden Straße erkannt werden. Auch wenn diese zur mittleren Linie gehören, sollen sie hier auf Grund des derzeitigen Regelungskonzepts noch nicht in die Karte aufgenommen werden. 
% Das ist freilich nicht falsch, aber für das aktuelle Regelungskonzept auch nicht zielführend. 

\paragraph{Verifikation}

Die Prozedur der Verifizierung verläuft nach einem vergleichbaren Prinzip wie der in Abschnitt~\ref{ssec:fahrspurerkennung:riverflow:randlinie} beschriebene Riverflow-Algorithmus. Dabei wird davon ausgegangen, dass sich das Fahrzeug größtenteils in der Fahrspur befindet, sodass mindestens ein Mittellinienstrich unweit vor dem Auto positioniert ist. Initial wird ein Suchfenster erstellt, welches in der Höhe den reichlichen Abstand zwischen den Mittelpunkten zweier benachbarter Mittelstreifen annimmt. Das erste sich in diesem Bereich aufhaltende Objekt wird dann als verifiziert angesehen, wenn seine Orientierung nicht übermäßig von der des Fahrzeugs abweicht. Ausgehend von diesem Punkt \glqq hangelt\grqq{} sich der Verifikations-Algorithmus von Strich zu Strich, indem er eine neue \gls{acr:roi} (ein neues Suchfenster) aufstellt. Da zu jedem (verifizierten) Punkt seine Orientierung bekannt ist, wird der nächste Mittellinienpunkt in dieser Richtung im ebenfalls bekannten Abstand zwischen zwei Punkten vermutet und dort der Mittelpunkt des neuen Suchfensters definiert. Dieses Schema wiederholt sich, bis entweder der Bildrand erreicht ist, keine Punkte im Suchfenster liegen, oder sich die Orientierung von einem zum nächsten Strich zu stark geändert hat.

Das gleiche Vorgehen wird zudem wahlweise vom ersten Punkt hinter dem Fahrzeug ausgehend wiederholt, sodass auch schon passierte Mittellinienpunkte, welche z.B. bei einem zukünftig implementierten Überholvorgang benötigt werden könnten, verifiziert sind.

\begin{figure}[ht]
	\centering
	\includegraphics[width=0.9\textwidth]{fahrspurerkennung_riverflow_regionprops.png}
	\caption{Detektion von Punktgruppen mit bestimmten Eigenschaften mithilfe der \emph{regionprops}-Funktion}
	\label{fig:riverflow:mittellinie:regionprops}
\end{figure}

In Abb.~\ref{fig:riverflow:mittellinie:regionprops} sehen wir das Resultat der Mittellinien-/Stricherkennung. Im gleichen Zug wird die \emph{regionprops}-Funktion auch zur Randstricherkennung bei Kreuzungen genutzt. Diese Informationen werden im Riverflow-Algorithmus verwendet, welcher im nun folgenden Abschnitt beschrieben ist.
\subsection{Randlinie}
\subsubsection{Startpunktgewinnung}
Um den eigentlichen Riverflow-Algorithmus ausführen zu können werden wird die ungefähre Lage der seitlichen Fahrbahnmarkierungen in der Nähe des Fahrzeugs benötigt. Zur Ermittlung dieser gibt es 3 Möglichkeiten:
\begin{enumerate}
\item \label{item:solidline:startpoints:dashedline}
Die Bestimmung durch Orientierung und Position des ersten vor dem Fahzeug gelegenen Mittellinienelementes. Hierzu wird der Mittelpunkt des Elementes entlang der Orientierung desselbigen um die Fahrspurbreite nach links/rechts verschoben. BILD
\item \label{item:solidline:startpoints:hough}
Eine eindimensionale Hough-Transformation des Bildausschnittes direkt vor dem Fahrzeug wie in REFERENZ beschrieben. BILD
\item \label{item:solidline:startpoints:fixed}
Annahme einer festen Position vor dem Fahrzeug/im Kamerabild.
\end{enumerate}

\subsubsection{zentraler Algorithmus}
Durch Nutzung der Eigenschaften \ref{item:riverflow:rule:solidline} und  \ref{item:riverflow:rule:curvature} kann ausgehend von der aktuellen Linienorientierung in einem bestimmten Kegel, dessen Öffnungswinkel durch die maximale Krümmung der Fahrspur definiert ist, der weitere Verlauf der Fahrbahnmarkierung vermutet werden. BILD
Algorithmisch genutzt wird dieses Wissen durch die Verwendung der aus \ref{ssec:fahrspurerkennung:kalman:messung} bekannten \glspl{glos:scanline}.
Zuerst wird ein Verschiebungsvektor \begin{math} \boldsymbol{v_n} \end{math} zwischen vorletztem (\begin{math} \boldsymbol{p_{n-1}} \end{math}) und letztem gefundenen Punkt  (\begin{math} \boldsymbol{p_n} \end{math})  der Fahrbahnmarkierung gebildet \eqref{eq:riverflow:solidline:dispvec}. Ausgehend vom letzten gefundenen Punkt der Linie (\begin{math} \boldsymbol{p_n} \end{math}) wird selbiger nun um den Vektor \begin{math} \boldsymbol{v_n} \end{math} weiterverschoben und bildet den Mittelpunkt  \begin{math} \boldsymbol{m_{n+1}}  \end{math} der nächsten \gls{glos:scanline} \eqref{eq:riverflow:solidline:scanlinemidpoint}.
\begin{equation}
\label{eq:riverflow:solidline:dispvec}
\boldsymbol{v_n} =  \boldsymbol{p_n} - \boldsymbol{p_{n-1}}
= 
\begin{pmatrix}
v_{x_n} \\
v_{y_n}
\end{pmatrix}
\end{equation}
\begin{equation}
\label{eq:riverflow:solidline:scanlinemidpoint}
\boldsymbol{m_{n+1}} =  \boldsymbol{p_n} + \boldsymbol{v_n}
\end{equation}
Mithilfe des zu \begin{math} \boldsymbol{v_n} \end{math} \eqref{eq:riverflow:solidline:dispvec} senkrechten Richtungsektors \begin{math} \boldsymbol{d_{n+1}} \end{math} \eqref{eq:riverflow:solidline:scanlinedirectionvec} der \begin{math} (n+1)\end{math}-sten  \gls{glos:scanline} \begin{math} \boldsymbol{s_{n+1}} \end{math} kann selbige wie folgt beschrieben werden:
\begin{equation}
\label{eq:riverflow:solidline:scanlinecontinous}
\boldsymbol{s_{n+1}} =
\boldsymbol{m_{n+1}}  + \alpha \cdot \boldsymbol{d_{n+1}}
\; \alpha \in \mathbb{N}
\end{equation}
\begin{equation}
\label{eq:riverflow:solidline:scanlinedirectionvec}
\boldsymbol{d_{n+1}} =
\begin{pmatrix}
-v_{y_n} \\
v_{x_n}
\end{pmatrix}
\end{equation}
Der Wertebereich des skalaren Faktors \begin{math} \alpha \end{math} gibt hierbei die Ausdehnung der \gls{glos:scanline} um Ihren Mittelpunkt \begin{math} \boldsymbol{m_{n+1}}  \end{math} herum an.
Da die Filterantwort des Kantendetekors durch den Algorithmus zur Erkennung der Mittellinie schon für das gesamte Bild vorliegt, kann zur Erkennung der seitlichen Fahrbahnmarkierung auf selbige zurückgegriffen werden. Hierfür wird die Scanline in eine diskrete Koordinatenserie überführt \ref{eq:riverflow:solidline:scanlinediscrete}, deren Elemente auf ganzzahlige Werte gerundet werden.
\begin{equation}
\label{eq:riverflow:solidline:scanlinediscrete}
\boldsymbol{s_{{n+1}_d}} =
\boldsymbol{m_{n+1}}  + z \cdot \boldsymbol{d_{n+1}} 
\; z \in \mathbb{Z}
\end{equation}
Die durch die Koordinatenserie adressierten Pixel der Filterantwort können nun zur weiteren Verarbeitung in einen Zeilenvektor geschrieben werden. 
 \begin{equation}
 \boldsymbol{f} =
 \begin{pmatrix}
 f_1 & f_2 &  \dots & f_i & \dots & f_n
 \end{pmatrix}
 \end{equation}
Nun wird  \begin{math} \gls{math:max}(\boldsymbol{f})  \end{math} ermittelt. Ist \begin{math} \gls{math:max}(\boldsymbol{f})  \end{math} größer als ein bestimmter Schwellwert \begin{math} s \end{math} , wird der Index \begin{math} i_{max} \end{math} des Maxima in \begin{math} \boldsymbol{f}  \end{math} vorgemerkt. Um eine Linie nicht wiederholt zu finden, muss die in \begin{math} \boldsymbol{f} \end{math} verbleibende, durch dieselbe Markierung verursachte Filterantwort entfernt  werden.
\begin{equation}
f_{i_{max}-\text{Linienbreite}/2}  \dots f_{i_{max}} 
 \dots  f_{i_{max}+\text{Linienbreite}/2} = 0
 \end{equation}
Darauffolgend kann nach weiteren \begin{math} s \end{math} überschreitendenden Einträgen in \begin{math} \boldsymbol{f} \end{math} gesucht und wie mit dem ersten \begin{math} \gls{math:max}(\boldsymbol{f})  \end{math} verfahren werden.
Wird kein ausreichend großes \begin{math} \gls{math:max}(\boldsymbol{f}) mehr gefunden, können den gefundenen Indizes \begin{math} i_{max} \end{math} via \eqref{eq:riverflow:solidline:scanlinediscrete} Koordinaten \begin{math} \boldsymbol{p_{n+1}} \end{math} zugeordnet werden GLEICHUNG.
Sind mehrere \begin{math} \boldsymbol{p_{n+1}} \end{math} vorhanden, werden ab diesem Zeitpunkt mehrere Hypothesen einer seitlichen Fahrbahnmarkierung verfolgt.
Jetzt kann die nächste Iteration des Riverflow-Algorithmus mit der Bildung des folgenden Verschiebungsvektors \boldsymbol{v_{n+1}} starten.

\subsubsection{Sonderfall 1. und 2. Iteration}
Da in der 1. und 2. Iteration keine Punkte \begin{math} \boldsymbol{p_{n-1}} \end{math} und \begin{math} \boldsymbol{p_n} \end{math}  vorhanden sind, muss der Verschiebungsvektor \begin{math} \boldsymbol{v} \end{math} anderweitig bestimmt werden. 
Wurde der Startpunkt wie in \ref{item:solidline:startpoints:dashedline} beschrieben durch ein Mittellinienelement bestimmt, kann dessen Orientierung als Richtung des Verschiebungsvektors genutzt werden.
In den beiden anderen Fällen \ref{item:solidline:startpoints:hough} und \label{item:solidline:startpoints:fixed} wird eine konstante Verschiebung in Fahrtrichtung des Fahrzeugs genutzt.

\subsubsection{Ende des Algorithmus}
Der Algorithmus wird beendet, sobald:
\begin{itemize}
\item Der Mittelpunkt \begin{math} \boldsymbol{m}  \end{math} der nächsten Scanline außerhalb des Bildbereichs liegt
\item Auf der aktuellen Scanline keine Maxima größer als der Schwellwert \begin{math} s \end{math} gefunden wurden.
\end{itemize}

\subsubsection{Sonderfall gestrichelte Randlinien}

\section{Verifikation \dcsecondauthorshort}
\label{ssec:fahrspurerkennung:riverflow:verifikation}
Nachdem für Mittel- und Seitenlinie repräsentative Punkte aus dem Bild extrahiert wurden, gilt es deren Sinnhaftigkeit zu überprüfen. Somit können:
\begin{enumerate}
\item Falsch erkannte Fahrbahnmarkierungselemente eliminiert werden.
\item Bei Vorhandensein mehrerer Hypothesen für die Seitenlinien die Besten ausgewählt werden.
\end{enumerate}

\subimport{verifikation/}{veri_mittellinie}
\subimport{verifikation/}{veri_randlinie}
