\subsection{Mittellinie}

Ähnlich wie im Kapitel~\ref{sec:maskenbau} beschrieben findet dieser Ansatz die zur Mittellinie zugehörigen Elemente, ohne auf richtig definierte \gls{acr:roi} angewiesen zu sein. Ausgangspunkt ist ebenfalls wieder das binarisierte Bild, auf das die Matlab-Funktion \emph{regionprops} angewendet wird. Diese bereits verfügbare Funktion findet als Alternative zum in ~\ref{sec:maskenbau} erwähnten \glqq Ringfilter\grqq{} alle zusammenhängende Pixelgruppen und extrahiert zusätzlich eine Vielzahl derer Eigenschaften, wie Länge, Breite, Orientierung und Mittelpunkt. Da wie in Punkt~\ref{item:riverflow:rule:dashedline} bereits erwähnt die mittlere Fahrbahnmarkierung genau spezifiziert ist, werden so alle potentiellen Mittellinienstriche vorgemerkt. Diese Vorgehensweise funktioniert allerdings nur dann gut, wenn die Güte des binarisierten Bildes gut genug ist, sodass Striche \gls{acr:zb} nicht unterbrochen sind. Trotz der guten Qualität des Bildes kommt es jedoch vor, dass falsche Bildteile als Mittelstriche erkannt und vorgemerkt werden. Deswegen stellen wir durch eine Verifizierung sicher, dass möglichst keine falschen Punkte in die Weltkarte, die später zur Regelung des Autos genutzt wird, eingetragen werden. Schon an der großen Vier-Straßen-Kreuzung kommt es sonst zum Problem, dass die Mittellinienstriche der kreuzenden Straße mit erkannt werden. Das ist freilich nicht falsch, aber für das aktuelle Regelungskonzept auch nicht zielführend. 

\paragraph{Verifikation}

Die Prozedur der Verifizierung verläuft nach einem vergleichbaren Prinzip wie der in~\ref{ssec:fahrspurerkennung:riverflow:randlinie} beschriebene Riverflow-Algorithmus. Dabei wird davon ausgegangen, dass sich das Fahrzeug halbwegs in der Fahrspur befindet, sodass sich mindestens ein Mittellinienstrich unweit vor dem Auto befindet. Es wird also initial in einem  Suchfenster, welches aus Parametern so eingestellt ist, dass es reichlich die Höhe eines Abstandes zwischen den Mittelpunkten zweier benachbarter Mittelstreifen annimmt, nach einem Mittelpunkt der vorgemerkten Mittelstriche gesucht. Das erste nahe zum Auto gelegene Objekt wird dann als verifiziert angesehen, wenn seine Orientierung nicht zu sehr von der des Fahrzeugs abweicht. Ausgehend von diesem Punkt \glqq hangelt\grqq{} sich der Verifikations-Algorithmus von Strich zu Strich, indem er eine neue \gls{acr:roi} (ein neues Suchfenster) aufstellt. Da zu jedem (verifizierten) Punkt seine Orientierung bekannt ist, wird der nächste Mittellinienpunkt in dieser Richtung im ebenfalls bekannten Abstand zwischen zwei Punkten vermutet und dort der Mittelpunkt des neuen Suchfensters definiert. Das Schema wiederholt sich, bis entweder der Bildrand erreicht ist und keine Punkte mehr im Suchfenster liegen, oder sich die Orientierung von einem zum nächsten Strich zu stark geändert hat.

Das gleiche Vorgehen wird wahlweise auch nach hinten wiederholt, sodass auch hinter dem Auto liegende Mittellinienpunkte, die \gls{acr:zb} beim Überholen benötigt werden könnten, verifiziert werden.

BILD