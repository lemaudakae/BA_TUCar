\chapter{Ansätze zur Fahrspurerkennung \dcsecondauthorshort}
\label{cha:fahrspurerkennung}
Ein zentrales Ziel dieser Arbeit stellt die Erkennung von Fahrspuren in einem Kamerabild dar. Mithilfe der im vorherigen Kapitel beschriebenen Bildvorverarbeitung wurde bereits eine Extraktion der für die folgenden Algorithmen notwendigen Eingangsinformationen vorgenommen. Die grundsätzliche Aufgabe der Konzepte zur Fahrbahndetektion besteht in einer weiteren Reduktion und Sortierung dieser vorverarbeiteten Daten zu einem mathematischen Modell der Straße. Die Komplexität dieses Modells variiert in der Literatur sehr stark \autocite{naroteReviewRecentAdvances2018}. Einige simple Repräsentationen wurden von Beginn der Recherche an ausgeschlossen, da sie dem vorgegebenen Szenario nicht gerecht werden: 
\begin{itemize}
\item Die deutliche Krümmung der Kurven unserer Teststrecke lässt eine Annäherung des Straßenverlaufs durch eine Gerade nicht zu. Ein komplexeres Modell (mehr als zwei Parameter) schließt die oft verwendete, fehlerresistente Hough-Transformation (s. Abschnitt~\ref{sec:grundlagen:hough}) weitgehend aus, da ein mehr als zweidimensionaler Modellraum zu rechenaufwendig für eine Echtzeitanwendung ist. 
\item Da auch eine Änderung der Krümmung (S-Kurve) des zu approximierenden Fahrspurverlaufs abgebildet werden soll, kann auch das oft verwendete Polynom zweiten Grades den Anforderungen nicht genügen. 
\end{itemize}
Die ersten zwei im Folgenden beschriebenen Verfahren bedienen sich deshalb eines Polynoms 3. Grades zur Modellierung der Fahrbahn bzw. deren Markierungen. Die final eingesetzte Herangehensweise nutzt eine Liste von Punkten, deren Anzahl in jedem bearbeiteten Bild unterschiedlich sein kann. Es wird somit kein immer mit derselben Anzahl Parameter beschreibbares Modell zugrunde gelegt. 

\subimport{ransac/}{ransac}
\subimport{kalman/}{kalman}
\subimport{riverflow/}{riverflow}
