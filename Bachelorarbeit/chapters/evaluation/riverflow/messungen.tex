\subsection{Messungen}
In diesem Abschnitt sollen charakteristische Messwerte der Fahrspurerkennung, Regelung, sowie dem Zusammenspiel beider Komponenten dargestellt und diskutiert werden. Dies soll unter Veränderung der Umgebungsbedingungen sowie mit unterschiedlichen Fortbewegungsgeschwindigkeiten des Modellfahrzeugs geschehen.
  
\subsubsection{Evaluation ohne Ground Truth}
Für die Erkennung von Fahrspuren auf Einzelbildern und die daraus resultierende Bewegung des Fahrzeugs existieren zum jetzigen Zeitpunkt keine Vergleichswerte. 

\paragraph{Bildverarbeitung} 
Für ein Benchmark der Fahrspurerkennung könnte ein Ground-Truth-Datensatz in Form von Bildern mit richtig eingetragen Fahrbahnmarkierungen die präziseste Bewertung hervorbringen. Da die Erstellung eines solchen jedoch sehr zeitaufwendig ist, wurde eine leichter automatisierbare Methode der Evaluation gewählt.

Durch die implementierte Verifikation der erkannten Koordinatenserien kann eine Fehlerkennung weitgehend ausgeschlossen werden. Den deutlichsten Indikator für ein falsch verarbeitetes Bild stellt also eine nicht oder nur auf einem sehr kurzen Abschnitt detektierte Fahrbahnmarkierung dar. Den Grenzwert der Länge für eine fehlerhaft erkannte Linie ist oft unkritisch, da der Detektionsalgorithmus in den meisten Fällen zufriedenstellend funktioniert oder schon bei kurzer Linienlänge fehlschlägt. Wir haben uns dafür entschieden etwa die halbe Sichtweite des Fahrzeugs (ab Stoßstange bis Bildrand ) als Schwellwert anzusehen.

\paragraph{Regelung} 
Die Güte der Regelung bzw. des Zusammenspiels aus Bildverarbeitung und Regelung könnte am exaktesten durch ein hinreichend genaues System zur Lokalisierung des Fahrzeugs im Testszenario ermittelt werden. Somit wäre es möglich die Abweichung der gefahrenen Strecke von einer vorgegebenen Ideallinie auszuwerten. Da die Inbetriebnahme des an der Professur vorhandenen zweidimensionalen Tracking-Systems viel Einarbeitungszeit benötigt, wird hier auf eine einfache Methode zurückgegriffen. Es wird lediglich vermerkt ob das Fahrzeug die Fahrspur völlig verlässt oder dem Straßenverlauf zufriedenstellend folgt.

\subsubsection{Ermittlung der Maximalgeschwindigkeit\&Parametertuning}
Dieser Paragraph beschäftigt sich mit der Ermittlung der maximal fahrbaren Geschwindigkeit bei Nutzung der entwickelten Bildverarbeitungs- und Regelungsstruktur. Durch die Auswertung verschiedener Messwerte soll der \glqq Flaschenhals\grqq\ gefunden und ,wenn möglich, durch Parameteranpassung beseitigt werden.

Wichtige, noch anzupassende Parameter stellen dar:
\begin{itemize}
\item Die Frequenz, in der Bilder verarbeitet werden
\item Die Frequenz, mit der die Regelung stattfindet
\item Die Entfernung des Zielpunktes der Regelung vom Fahrzeug
\end{itemize}

Den Ausganspunkt für die folgenden Optimierungen bildet eine Geschwindigkeit von \(0.1 \frac{m}{s}\). Die Frequenz, mit welcher die Bilder verarbeitet werden wurde auf \(1 Hz\) festgelegt, da die Extraktion der Linienpunkte eines Testbildes ca. \(0,5 s\) benötigt. Die Frequenz der Regelung wurde initial auf \(20 Hz\) festgelegt, da der Rechenaufwand für einen Regelungszyklus sehr gering ist. Eine noch höhere Regelfrequenz verspricht keinen Performancegewinn und läge schon nahe der maximalen Ansteuerfrequenz des Lenkservos sowie der Abtastrate der \gls{acr:imu}.




\subsubsection{Leistungsfähigkeit bei unterschiedlichen Geschwindigkeiten}
In diesem Abschnitt soll untersucht werden, ob eine Reduzierung der Geschwindigkeit positive Auswirkungen auf die Zuverlässigkeit von Bildverarbeitung und Regelung besitzt.

\subsubsection{Leistungsfähigkeit bei Veränderung der Umgebungsbedingungen - Störobjekte}
Da geplant ist die Teststrecke in Zukunft realistischer zu gestalten, d.h. Objekte wie Straßenschilder, andere Fahrzeuge, Personen und Gebäude hinzuzufügen. Der Einfluss solcher \glqq Störobjekte\grqq\ auf den Fahrspurverfolgungsalgorithmus soll nun untersucht werden.

\paragraph{Menge der Objekte in einem Bild}
Zu Beginn soll lediglich die Menge der \glqq Störobjekte\grqq\, welche sich in einem Bild befinden, variiert werden. Da sich die Objekte hinreichend weit von der Fahrbahnmarkierung entfernt befinden, ist kein Einfluss die Bildverarbeitung zu erwarten.

\paragraph{Entfernung der Objekte von den Fahrbahnmarkierungen}
Verringert man die Distanz zwischen \glqq Störobjekten\grqq\ und der Straße, wird es wahrscheinlicher, dass Kanten dieser Gegenstände als Fahrbahnmarkierung erkannt werden.

\paragraph{Kontrast der Objekte im Vergleich zum Testszenario}
Ist der Helligkeitsunterschied zwischen Fahrbahn und \glqq Störobjekt\grqq\ bzw. der leeren Fläche neben der Fahrbahn und dem \glqq Störobjekt\grqq\ sehr gering, so sollte keine zusätzliche Kante im Bild gefunden und als Fahrbahnmarkierung identifiziert werden . Diese These soll im folgenden Test bestätigt betrachtet werden.

\subsubsection{Leistungsfähigkeit bei Veränderung der Umgebungsbedingungen - fehlende Fahrbahnmarkierungen}
Es soll gezeigt werden, dass auch bei fehlen bestimmter Fahrbahnmarkierungen eine Verfolgung des Straßenverlaufs möglich ist.






