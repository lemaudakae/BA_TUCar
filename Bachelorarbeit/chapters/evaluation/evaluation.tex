\chapter{Evaluation \dcfirstauthorshort}
\label{cha:evaluation}

In diesem Kapitel wird das final zur Fahrspurverfolgung genutzte Vorgehen ausführlich bewertet. Zu Beginn wird auf prinzipielle Vor- und Nachteile eingegangen, um schließlich die Performanz anhand einiger Tests zu demonstrieren. Neben einer zeitlichen Betrachtung wird die Robustheit des Riverflow-Algorithmus bezüglich Geschwindigkeit, Störobjekten und abgedeckten Bereichen in der Fahrspur getestet.

Abschnitt \ref{ssec:evaluation:messungen:ohnegt} beleuchtet die Schwierigkeiten, ein entwickeltes System ohne Vergleichswerte oder Messsysteme objektiv zu bewerten. Anschließend sollen charakteristische Messwerte der Fahrspurerkennung, Regelung, sowie dem Zusammenspiel beider Komponenten dargestellt und diskutiert werden. In Paragraph \ref{ssec:evaluation:messungen:laufzeit} wird die Laufzeit der Fahrspurverfolgungskomponenten ermittelt und auf Basis dieser Messwerte die maximal verarbeitbare Bildfrequenz eingestellt. Passus \ref{ssec:evaluation:messungen:weltkarte} vergleicht die erzeugte Weltkarte mit der Zeichnung des Testszenarios um Rückschlüsse auf die Qualität der am Kartenaufbau beteiligten Messungen zu ziehen. In Passage \ref{ssec:evaluation:messungen:geschwindigkeit} soll das Verhalten des Modellautos bei unterschiedlichen Geschwindigkeiten evaluiert werden. 

Die anknüpfend folgenden Abschnitte stellen eine größtenteils qualitative Evaluation des Fahrverhaltens bei Veränderung des Testszenarios dar. Da fortan das Straßenszenario vielseitiger genutzt werden soll, ist es denkbar, dass sich künftig zusätzliche Objekte im Sichtbereich des Roboters befinden. Daher untersucht Absatz \ref{ssec:evaluation:messungen:objekte_hinzufuegen} die Reaktion der Fahrspurverfolgung auf das Hinzufügen von Objekten zur Modellwelt, während in Passage \ref{ssec:evaluation:messungen:weniger_infos} die Robustheit bei Reduktion der Kennzeichen einer Straße betrachtet wird.

\paragraph{Überlegungen zur Darstellung der Messwerte  \dcsecondauthorshort}
Um möglichst viele Informationen kompakt abzubilden, wurden unter anderem die Diagrammtypen Boxplot sowie Histogramm genutzt. Da kurz nach dem Start des Fahrzeugs 
\begin{itemize}
	\item
	MATLAB viele Funktionen erst noch in den Hauptspeicher laden muss (\glqq Warmlaufphase\grqq)
	\item
	Das Fahrzeug vor vollendeter Verarbeitung des ersten Bildes schon ohne gültigen Zielpunkt losfährt
\end{itemize}
wurden diese nicht-repräsentativen Phasen in entsprechenden Grafiken ausgelassen, da sie die Aussagekraft der Messwerte beeinträchtigen würden. 
%In der Bildunterschrift wurde jeweils das Stichwort \glqq ohne Startphase\grqq\ eingefügt.

\subimport{diskussion/}{diskussion_prinzip}
\subimport{messungen/}{messungen}

