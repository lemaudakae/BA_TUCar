\section{Aufbau der Arbeit \dcfirstauthorshort}

Der folgende Inhalt der Ausarbeitung ist ähnlich der Entstehungsreihenfolge aufgebaut. Anfangs finden sich neben der Vorstellung der physikalisch greifbaren Betriebsmittel (Kapitel~\ref{cha:fahrzeug_szenario}) grundlegende Beschreibungen der genutzten Methoden der Bildverarbeitung und Mathematik (Kapitel~\ref{cha:grundlagen}). Die Verarbeitung der Bilder und die Berechnungen für die Steuerung des Autos führen wir auf einem externen Rechner mit MATLAB durch. Wie die Software dazu aufgebaut ist, wird im anschließenden Kapitel~\ref{cha:software_struktur} erläutert. Der danach folgende Hauptteil befasst sich mit der Bildverarbeitung (Kapitel~\ref{cha:bildvorverarbeitung} und~\ref{cha:fahrspurerkennung}) und Trajektorienplanung (Kapitel~\ref{cha:regelung}). Da wir uns im Vorfeld nicht auf einen speziellen Algorithmus für die Fahrspurverfolgung festgelegt haben und einen möglichst gut funktionierenden Ansatz finden wollten, stellen wir drei bearbeitete Methoden zur Fahrspurerkennung vor (Abschnitte~\ref{sec:maskenbau}, \ref{sec:fahrspurerkennung_kalman}, \ref{sec:fahrspurerkennung:riverflow}), auch wenn letztlich nur eine Vorgehensweise lauffähig implementiert wurde. Nach einer umfassenden Evaluation (Kapitel~\ref{cha:evaluation}) blicken wir zum Abschluss auf mögliche Aufgaben, die man zukünftig mit dem Modellfahrzeug in dem Straßenverkehrsszenario bewältigen könnte (Kapitel~\ref{cha:ausblick}).

\subsubsection{Aufteilung \dcsecondauthorshort}

Diese Arbeit ist eine Gruppenarbeit.
Während der Einarbeitungsphase in grundlegende Methoden der Bildverarbeitung, dem Kerngebiet dieser Arbeit, wurden Problematiken vorrangig zusammen erörtert. Nach dem individuellen Auseinandersetzen mit vielen Dokumenten zu ähnlichen Projekten des Themengebiets \glqq Autonomes Fahren mit Modellfahrzeugen\grqq{} ergaben sich viele Einzelkomponenten, die zu einer lauffähigen Fahrspurverfolgung nach unseren Vorgaben notwendig sind. Bei der Implementierung dieser Verfahren beschäftigte sich je ein Autor vorrangig mit dem Sachverhalt. Die verschiedenen Programmbausteine sind dennoch oft durch die Überlegungen beider Gruppenmitglieder entstanden, da Ideen oft gegenseitig optimiert und auf Übergabeparameter anderer Programmteile abgestimmt wurden. Nach Abschluss des praktischen Teils der Arbeit verfasste der Ideengeber eines jeden Elements der Software das entsprechende Kapitel.   
Aufgrund dieses Gruppenarbeitscharakters soll kenntlich gemacht werden, wer Urheber eines jeden Abschnitts ist. Die Markierung erfolgt durch \dcfirstauthorshort\ für \dcfirstauthorfirstname\ \dcfirstauthorlastname\  und \dcsecondauthorshort\  für \dcsecondauthorfirstname\ \dcsecondauthorlastname{}. Das zuletzt gegebene Label zeigt den Autor des aktuellen Textes an, bis das jeweils andere Symbol einer Überschrift angefügt ist.


