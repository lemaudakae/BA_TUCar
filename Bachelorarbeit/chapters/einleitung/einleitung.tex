\chapter{Einleitung \dcsecondauthorshort}

Autonomes Fahren ist schon längst kein Sciencefiction mehr. Prototypen autonomer Fahrzeuge existieren schon seit geraumer Zeit, ein geschichtlicher Abriss der Anfänge kann unter anderem in \autocite[41-67]{kroegerAutomatisierteFahrenIm2015} nachgeschlagen werden. Doch gerade in den letzten Jahren hielten Funktionalitäten autonomer, mobiler Roboter in Form von Fahrerassistenzsystemen Einzug in Serienfahrzeugen. Besonders die Erkennung und Verfolgung von Fahrspuren stellt dabei eine wiederkehrende Aufgabe dar, aktuelle Forschungen \autocite{kunzeReadingLanesRoad2018}, sowie Vergleiche vorhandener Methoden \autocite{naroteReviewRecentAdvances2018} zeigen die Relevanz dieses Themas. Die Implementierung einer Fahrspurdetektion sowie Regelung auf Basis der gewonnenen Informationen stellt den Hauptbestandteil dieser Arbeit dar.

%Um Studierenden der Technischen Universität Chemnitz einen Einblick in die Welt der führerlosen Automobile zu bieten, wurde im Kontext dieser Arbeit eine Modellumgebung samt Fahrzeug entwickelt. Die entstandene Hard- und Software soll in kommenden Jahren bei Praktikumsaufgaben zur Vorlesung \glqq Autonome Systeme\grqq{}, weiteren Lehrveranstaltungen sowie diversen studentischen Arbeiten zum Einsatz kommen. 

\subimport{zielstellung/}{zielstellung}
\subimport{wettbewerbe/}{wettbewerbe}
\subimport{aufbau/}{aufbau}