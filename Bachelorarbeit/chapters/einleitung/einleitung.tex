\chapter{Einleitung \dcsecondauthorshort}

Autonomes Fahren ist schon längst kein Sciencefiction mehr. Prototypen autonomer Fahrzeuge existieren schon seit geraumer Zeit \autocite{kroegerAutomatisierteFahrenIm2015}. Doch gerade in den letzten Jahren hielten Funktionalitäten autonomer, mobiler Roboter in Form von Fahrerassistenzsystemen Einzug in Serienfahrzeugen. Besonders die Erkennung und Verfolgung von Fahrspuren stellt dabei eine wiederkehrende Aufgabe dar \autocite{kunzeReadingLanesRoad2018, naroteReviewRecentAdvances2018} und kann als ein Hauptbestandteil dieser Arbeit angesehen werden. 

Um auch Studierenden der Technischen Universität Chemnitz einen Einblick in die Welt der führerlosen Automobile zu bieten, wurde im Kontext dieser Arbeit eine Modellumgebung samt Fahrzeug entwickelt. Die entstandene Hard- und Software soll in kommenden Jahren bei Praktikumsaufgaben zur Vorlesung \glqq Autonome Systeme\grqq{}, weiteren Lehrveranstaltungen sowie diversen studentischen Arbeiten zum Einsatz kommen. 

\subimport{zielstellung/}{zielstellung}
\subimport{wettbewerbe/}{wettbewerbe}
\subimport{aufbau/}{aufbau}