\section{Hintergrund und Ziel}

Der Anstoß dieses Projektes ist die Idee, in Zukunft einen weiteren Praktikumsversuch aufzubauen. An der Professur Prozessautomatisierung der TU Chemnitz besteht bereits ein Projektpraktikum, in dem mit eigens dafür entwickelten Robotern ein schachbrettartiges Labyrinth autonom durchfahren werden muss. Diese \glqq TUCbots\grqq{} besitzen allerlei Sensoren wie Encoder, Infrarot-Abstandsdetektoren und Anschlagsschalter, jedoch keine Kamera. Das Ziel ist nun, einen weiterführenden, neuen Praktikumsaufbau zu schaffen, in dem sich ein Modellfahrzeug mit Ackermannlenkung einzig mithilfe einer omnidirektionalen Kamera zurecht finden soll. 

Mit diesem Hintergrund bestand unsere Absicht darin, einen Fahrspurverfolgungsalgorithmus zu implementieren, welcher durch mögliche spätere Aufgabenstellungen für das Praktikum erweiterbar sein sollte. Dementsprechend mit dieser Arbeit verbundene Ziele sind folgend genannt:

\begin{itemize}
\item Entwurf einer geeigneten Teststrecke
\item Entzerren festgelegter Bildausschnitte einer omnidirektionalen Kamera
\item Filtern von Bildern
\item Linienerkennung in einem verarbeiteten Bild und damit verbundene mathematische Approximation der Fahrbahnmarkierungen
\item aktuelle Pose des Autos in Bezug eines globalen Koordinatensystems bestimmen
\item Aufbau einer Karte zur Pfadplanung
\item Trajektorienverfolgung durch Regelung des Lenkwinkels
\end{itemize}

Damit das Ergebnis der Ausarbeitung kein bloßes, auf das aktuelle Bild reaktives Fahrverhalten ist, stand das Eintragen von Informationen in eine Weltkarte während des Linienerkennungsprozesses von Anfang an fest. 

Die initiale, hauptsächliche Inspirationsquelle zur Themenwahl des vorliegenden Projektes stellten jedoch schon vorhandene, unter studentischen Teilnehmern ausgetragene Turniere mit breitem Aufgabenspektrum im Bereich führerloser Automobile dar, über die im folgenden Abschnitt berichtet werden soll.
