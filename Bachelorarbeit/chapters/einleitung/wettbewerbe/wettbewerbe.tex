\section{Studentische Wettbewerbe \dcsecondauthorshort}
Um Studierenden möglichst früh einen Einblick in die Welt selbstständig fahrender Automobile zu bieten, wurden in der Vergangenheit eine Vielzahl von Wettbewerben ausgetragen. Wie auch in dieser Arbeit bedient man sich hier oft eines Modellfahrzeugs, um die Kosten überschaubar zu halten und die Testumgebung kontrollierter gestalten zu können. Im Folgenden möchten wir einige dieser Wettbewerbe kurz vorstellen.
\subsubsection{Carolo-Cup}
Seit 2008 wird an der Technischen Universität Braunschweig der Carolo-Cup\footnote{https://wiki.ifr.ing.tu-bs.de/carolocup/carolo-cup} ausgetragen. Neben den sogenannten dynamischen Disziplinen
\begin{itemize}
\item Rundkurs ohne Hindernisse
\item Rundkurs mit Hindernissen
\item Einparken
\end{itemize}
welche mittels eines durch die Teams entwickelten Fahrzeugs im Maßstab 1:10 bestmöglich bewältigt werden sollen, wird in den statischen Disziplinen
\begin{itemize}
\item Präsentation und Konzept
\item Technische Ansätze
\end{itemize}
auf die Soft-Skills der Teammitglieder sowie die Innovation und ökonomische Bilanz beim Bau des Fahrzeugs Wert gelegt. Das Regelwerk des Carolo-Cups gab beim Entwurf der Testumgebung für das in dieser Arbeit verwendete Fahrzeug wichtige Anhaltspunkte, auch unsere Teststrecke bildet einen zweispurigen Rundkurs und kann zum Absolvieren verschiedener Sonderaufgaben wie Einparken, Verhalten an einer Kreuzung und Einbiegen in eine Nebenstraße verwendet werden.

\subsubsection{Audi Autonomous Driving Cup}
Auch der Audi Autonomous Driving Cup\footnote{https://www.audi-autonomous-driving-cup.com/} bot Inspiration für die folgenden Kapitel. Im Gegensatz zum Carolo-Cup wird bei diesem Turnier die Hardware samt eines Software-Frameworks vom Veranstalter gestellt. Somit verlagert sich der Schwerpunkt noch weiter in Richtung der Findung\&Implementierung schneller\&robuster Algorithmen. 
%Weiterhin wurden Konzepte der vorgegebenen Fahrzeuge auch im Rahmen anderer Projekte genutzt \autocite{jenschmischekDigitaleBildverarbeitungAuf2016}.

\subsubsection{Weitere Wettbewerbe}
Als weitere studentische Wettbewerbe, auf die wir erst im späteren Verlauf der Arbeit aufmerksam wurden, sind zu nennen:
\paragraph{NXP-Cup}
Auch beim NXP-Cup\footnote{https://community.nxp.com/groups/tfc-emea} werden weite Teile der Hardware vorgegeben. Die Implementierung der Algorithmen auf einem wenig performanten Mikrocontroller stellt eine besondere Herausforderung dar. 
\paragraph{AI Driving Olympics}
Die \textbf{A}rtificial \textbf{I}ntelligence Driving Olympics\footnote{https://www.duckietown.org/research/ai-driving-olympics} beschäftigen sich mit Künstlicher Intelligenz im Kontext autonomes Fahren, eine Kombination die durch moderne Rechentechnik klassischen Ansätzen oft überlegen ist. 