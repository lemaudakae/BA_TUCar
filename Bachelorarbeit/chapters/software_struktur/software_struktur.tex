\chapter{Struktur der Software}

\section{ROS}
Da der Austausch von Bilddaten und Ansteuerbefehlen zwischen Modellfahrzeug und PC via \gls{acr:ros} \autocite{ROSOrgPowering} abläuft, soll kurz darauf eingegangen werden. 
\begin{quotation}
\gls{acr:ros} ist eine Sammlung von Werkzeugen, Bibliotheken und Konventionen zur Vereinfachung der Entwicklung komplexen und robusten Roboterverhaltens für eine Vielzahl an Roboterplattformen. \autocite{ROSOrgROS}
\end{quotation}

\subsection{\glspl{glos:node}}
Ein \gls{acr:ros}-System besteht aus mehreren Programmen \glqq \glspl{glos:node}\grqq, welche über bestimmte Kommunikationskanäle \glqq \glspl{glos:topic}\grqq\ Datenpakete \glqq \glspl{glos:message}\grqq\ austauschen. Diese Applikationen können Information auf einem \gls{glos:topic} zur Verfügung stellen, sie fungieren als sogenannte \gls{glos:publisher}, oder nutzen, in diesem Fall man nennt sie \gls{glos:subscriber}.

\subsubsection{Fahrzeug}
\paragraph{Kamera-\gls{glos:node}}
Der Kamera-\gls{glos:node} veröffentlicht die Bilder als Rohdaten-\gls{glos:topic} sowie in komprimierter Form.
\paragraph{Fahrzeug-\gls{glos:node}}
Der Fahrzeug-\gls{glos:node} veröffentlicht:
\begin{itemize}
\item Odometriedaten 
\item Messwerte der \gls{acr:imu}
\item sonstige Informationen zum Fahrzeug (z.B. Akkuspannung)
\end{itemize}
und nimmt im Ansteuer-\gls{glos:topic} die gewünschte Geschwindigkeit und den Lenkwinkel entgegen.
\subsubsection{PC}
Auf dem PC wird nur der von \gls{glos:matlab} automatisch initialisierte \gls{glos:node} genutzt. Dieser abonniert das komprimierte Bilddaten-\gls{glos:topic} und veröffentlicht auf das Ansteuer-\gls{glos:topic} des Fahrzeugs.

\section{MATLAB}
Die im Hauptteil dieser Arbeit beschriebenen Algorithmen zur Fahrspurverfolgung wurden ausschließlich in \gls{glos:matlab} implementiert. Da die grundlegende Softwarestruktur in allen drei Fällen gleich ist, soll diese nun dargelegt werden.

\subsection{Initialisierung}
Ist das Fahrzeug betriebsbereit, d.h. werden alle \glspl{glos:node} des Fahrzeugs ausgeführt, so kann in \gls{glos:matlab} das Programm zur Fahrspurverfolgung aufgerufen werden. Hierfür wird die Initialisierungsroutine gestartet welche:
\begin{itemize}
\item die globalen Parameter lädt
\item die Weltkarte und die Pose initialisiert
\item den \gls{glos:matlab}-\gls{acr:ros}-\gls{glos:node} startet
\item die \gls{glos:subscriber} des Bilddaten-, Odometrie- und \gls{acr:imu}-\glspl{glos:topic} anlegt
\item eine konstante tangentiale Geschwindigkeit des Fahrzeugs vorgibt 
\end{itemize}

\subsection{Callbacks}
\paragraph{Odometrie-\gls{glos:callback}}
Das Fahrzeug veröffentlicht in konfigurierbaren Abständen seine aktuellen Odometriedaten.
Das Eintreffen eines solchen Datenpaketes in \gls{glos:matlab} führt zum Aufrufen der Odometrie-\gls{glos:callback}-Funktion. Die Hauptaufgabe dieses Programmteils besteht in der Regelung des Fahrzeugs, deren Ablauf im folgenden kurz erläutert wird.
\begin{enumerate}
\item Mithilfe der empfangenen, seit der letzten Odometrie-\gls{glos:message} verstrichenen Zeit, der in dieser Spanne ermittelten Durchschnittsgeschwindigkeit und der durch die \gls{acr:imu} erhaltenen Orientierung kann die Pose aktualisiert werden.
\item Auf Basis der neuen Pose kann mithilfe aller in der Weltkarte vorhandenen Punkte ein neuer Lenkwinkel bestimmt werden.
\end{enumerate}
\paragraph{Bild-\gls{glos:callback}}
Sobald ein neues Bild im zugehörigen \gls{glos:topic} verfügbar ist, wird ein Algorithmus zur Extraktion von Punkten der Fahrbahnmarkierung gerufen. Dieser fügt die gefundenen Koordinatenserien in die Weltkarte ein und macht sie so für die Regelung abrufbar.
