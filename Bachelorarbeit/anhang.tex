\appendix
\addcontentsline{toc}{chapter}{Anhang} % Zeigt "Anhang'' in Inhaltsverzeichnis
\newcommand{\fdname}[1]{\emph{#1}}
\chapter{Daten-DVD}	
\section{Digitale Version der Arbeit (PDF-Format)}
Die Bachelorarbeit befindet sich als PDF-Dokument \fdname{Bachelorarbeit} im Wurzelverzeichnis der Daten-DVD.

\section{Quellcode}
In den Ordnern \fdname{Polynombasierte Fahrspurerkennung} und \fdname{Punktbasierte Fahrspurerkennung} befindet sich der Quellcode zu den in Kapiteln \ref{cha:fahrspurerkennung} und \ref{sec:fahrspurerkennung:riverflow} differenzierten Fahrspurerkennungsmethoden. Da die grundsätzliche Vorgehensweise bei der Inbetriebnahme gleich ist, wurde lediglich ein PDF-Dokument \fdname{Anleitung} im Wurzelverzeichnis der DVD erstellt.

\subsection{Polynombasierte Fahrspurerkennung}
In den Verzeichnissen \fdname{Polynombasierte Fahrspurerkennung/Ransac} und \fdname{Polynombasierte Fahrspurerkennung/Kalman} ist der letzte Entwicklungsstand der in Abschnitt \ref{sec:polynombasierte_fahrspurerkennung:ransac} sowie \ref{sec:fahrspurerkennung_kalman} dargelegten, jedoch letztendlich verworfenen Fahrspurdetektionskonzepten, festgehalten. Da nie eine selbstständige Fahrt im Testszenario realisiert wurde, kann mit diesem Code lediglich der Linienverlauf in Einzelbildern extrahiert, sowie eine Weltkarte erstellt werden. Dies kann durch händische Steuerung des Fahrzeugs, sowie auf Basis der im Unterordner \fdname{Punktbasierte Fahrspurerkennung/ROSBags} befindlichen
ROS-Bag-Dateien geschehen.

\subsection{Punktbasierte Fahrspurerkennung}
\label{anhang:quellcode:punktbasierte}
Im Verzeichnis \fdname{Punktbasierte Fahrspurerkennung} ist der final genutzte Ansatz zur Fahrspurerkennung inklusive funktionstüchtiger Regelung zu finden. Hier befindet sich auch ein Ordner \fdname{Debug\_Daten}, welcher die Datenbasis des Kapitels \ref{cha:evaluation} enthält.

\section{gesicherte Internetquellen}
Im Ordner \fdname{gesicherte Internetquellen} sind alle referenzierten Websites gespeichert.
